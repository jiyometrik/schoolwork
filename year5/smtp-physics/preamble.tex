\setlength{\headheight}{14.49998pt} % make latex stop complaining!
\setcounter{tocdepth}{0}

% simpler commands
\renewcommand{\bf}[1]{\textbf{#1}}
\renewcommand{\it}[1]{\textit{#1}}
\renewcommand{\tt}[1]{\texttt{#1}}
\newcommand{\term}[1]{{\color{razz}\scshape#1}}
\newcommand{\code}[1]{{\color{razz}\ttfamily#1}}

% math macros
\newcommand{\mv}[1]{\ensuremath{\symbf{#1}}}
\newcommand{\f}[2]{\frac{#1}{#2}}
\newcommand{\df}[2]{\dfrac{#1}{#2}}
\newcommand{\lf}[2]{\slashfrac[auto]{#1}{#2}}
\newcommand{\reals}{\symbb{R}}
\newcommand{\rationals}{\symbb{Q}}
\newcommand{\ints}{\symbb{Z}}

\newcommand{\ii}{\mv{\hat{i}}}
\newcommand{\jj}{\mv{\hat{j}}}
\newcommand{\kk}{\mv{\hat{k}}}

% define accent colour used throughout the document
\definecolor{razz}{HTML}{DB5375}
\definecolor{ncs}{HTML}{26547C}
\definecolor{sung}{HTML}{ffd166}

% wrap figures
\usepackage{wrapfig}

% maths
\usepackage{derivative, amsthm, mathtools, amsmath}
\usepackage{cancel}

% expandable brackets
\usepackage{physics2}
\usephysicsmodule[tightbraces=true]{ab}

% font
\usepackage{fontspec}
\defaultfontfeatures{Scale=MatchLowercase}
\usepackage[
	math-style=TeX,
	warnings-off={mathtools-colon,mathtools-overbracket}
]{unicode-math}
\setmainfont{xcharter}
\setsansfont{tex gyre heros}
\setmonofont{tex gyre cursor}
\setmathfont{xcharter math}

% si units
\usepackage{siunitx}
\sisetup{
	mode=math,
	propagate-math-font=false,
	reset-math-version=true,
	reset-text-family=false,
	reset-text-series=false,
	text-family-to-math=false,
	text-series-to-math=false
}

% drawings
\usepackage{tikz}
\usetikzlibrary{arrows.meta}
\usepackage{annotate-equations}

% fun highlights
\usepackage[most]{tcolorbox}
\tcbuselibrary{breakable,theorems}
\newtcbox{\hil}[1][razz]{
	on line,
	arc=0pt,outer arc=0pt,
	colback=#1!25!white,
	boxsep=0pt,left=1pt,right=1pt,top=2pt,bottom=2pt,
	boxrule=0pt,
	breakable,enhanced jigsaw,
}

% make sure align*, aligned environments can stretch across pages
\allowdisplaybreaks

% numbering for equations, figures and tables etc
\numberwithin{equation}{section}
\numberwithin{table}{section}
\numberwithin{figure}{section}
\newtheorem{problem}{\hil{Problem}}[section]
