\setlength{\headheight}{14.49998pt} % make latex stop complaining!
\setcounter{tocdepth}{0}

% simpler commands
\renewcommand{\bf}[1]{\textbf{#1}}
\renewcommand{\it}[1]{\textit{#1}}
\renewcommand{\tt}[1]{\texttt{#1}}
\newcommand{\term}[1]{{\color{razz}\scshape#1}}
\newcommand{\code}[1]{{\color{razz}\ttfamily#1}}

% math macros
\newcommand{\mv}[1]{\ensuremath{\symbf{#1}}}
\newcommand{\f}[2]{\frac{#1}{#2}}
\newcommand{\df}[2]{\dfrac{#1}{#2}}
\newcommand{\lf}[2]{\slashfrac[auto]{#1}{#2}}
\newcommand{\reals}{\symbb{R}}
\newcommand{\rationals}{\symbb{Q}}
\newcommand{\ints}{\symbb{Z}}

\newcommand{\ii}{\mv{\hat{i}}}
\newcommand{\jj}{\mv{\hat{j}}}
\newcommand{\kk}{\mv{\hat{k}}}

% define accent colour used throughout the document
\definecolor{razz}{HTML}{DB5375}
\definecolor{ncs}{HTML}{26547C}
\definecolor{sung}{HTML}{ffd166}

% wrap figures
\usepackage{wrapfig}

% maths
\usepackage{derivative, amsthm, mathtools, amsmath}
\usepackage{cancel}

% expandable brackets
\usepackage{physics2}
\usephysicsmodule[tightbraces=true]{ab}

% font
\usepackage{fontspec}
\defaultfontfeatures{Scale=MatchLowercase}
\usepackage[
	math-style=TeX,
	warnings-off={mathtools-colon,mathtools-overbracket}
]{unicode-math}
\setmainfont{charis sil}
\setsansfont{fira sans}
\setmonofont{tex gyre cursor}
\setmathfont{xcharter math}

% si units
\usepackage{siunitx}
\sisetup{
	mode=math,
	propagate-math-font=false,
	reset-math-version=true,
	reset-text-family=false,
	reset-text-series=false,
	text-family-to-math=false,
	text-series-to-math=false
}

% drawings
\RequirePackage{pgfplots}% Also loads Tikz

% For advanced table calcs, e.g. 'y expr' to mathematically process input data:
\RequirePackage{pgfplotstable}

\RequirePackage[edges]{forest}% Comprehensive trees

\RequirePackage{tikz-3dplot}

% Only use libraries as needed to keep compilation times low
\usetikzlibrary{%
	positioning,% Relative positioning etc.
	calc,% Calculate distances, coordinates etc.
	% shapes, % For cross out
	% backgrounds,% Draw on background layer
	fit,% Fit new node around existing coordinates
	decorations,
	arrows,
	arrows.spaced,
	intersections,
	% trees,
	% circuits.ee.IEC,% Electrical engineering circuits lib
	patterns,
	% 3d,
	tikzmark,% Marks/coordinates at arbitrary positions
}%
\usepgfplotslibrary{%
	colorbrewer,%
	units,%
	dateplot,
	fillbetween,
	groupplots,
}%

\pgfplotscreateplotcyclelist{mod colorcycle}{% Modify existing colorbrewer
	{razz},
	{razz!25!white},
	{black!50},
	% {rdylbu5}% No comma or space here!
}%

\pgfplotscreateplotcyclelist{mod mark list}{% Modify predefined to start without mark
every mark/.append style={solid,fill=\pgfplotsmarklistfill}\\
every mark/.append style={solid,fill=\pgfplotsmarklistfill},mark=*\\
every mark/.append style={solid,fill=\pgfplotsmarklistfill},mark=square*\\
every mark/.append style={solid,fill=\pgfplotsmarklistfill},mark=triangle*\\
every mark/.append style={solid},mark=star\\
every mark/.append style={solid,fill=\pgfplotsmarklistfill},mark=diamond*\\
every mark/.append style={solid,fill=\pgfplotsmarklistfill!40},mark=otimes*\\
every mark/.append style={solid},mark=|\\
every mark/.append style={solid,fill=\pgfplotsmarklistfill},mark=pentagon*\\
}%

%%%%%%%%%%%%%%%%%%%%%%%%%%%%%%%%%%%%%%%%%%%%%%%%%%%%%%%%
% Global plot settings
%%%%%%%%%%%%%%%%%%%%%%%%%%%%%%%%%%%%%%%%%%%%%%%%%%%%%%%%
\pgfplotstableset{col sep=comma}% If ALL files/tables are comma-separated

\pgfplotsset{%
	% Version this document was originally created with;
	% see also https://tex.stackexchange.com/a/81912/120853
	% Important for reproducibility
	compat=1.16,
	% In case nodes reach out of plot area, dont clip them off.
	% See: https://tex.stackexchange.com/a/127904/120853 and
	% https://tex.stackexchange.com/a/311194/120853
	clip mode=individual,
	label style={font=\small},% https://tex.stackexchange.com/a/300673/120853
	tick label style={%
			font=\small,%
			/pgf/number format/1000 sep={\,}% Small space instead of comma
		},
	cycle multi list={%
			mod mark list\nextlist% Just in case; likely never reached
			linestyles*\nextlist
			% Iterate through last one first; if at end,
			% start anew using next in parent (linestyles *):
			mod colorcycle%
		},%
	colormap/viridis,% For 3D/surf plots
	legend style={%
			at={(0.5,1.05)},% Center hor. (0.5), slightly out of drawing(>1 vert.)
			anchor=south,%
			legend columns=-1,% -1 means only rows, as many columns as required
			font=\footnotesize,%
			draw=none,% Do not draw box
			fill=none,% No white fill. Important for gray backgrounds
		},%
	% These comment lines (visual separation) are super important.
	% Empty lines will throw errors here!
	unit code/.code 2 args={\unit{#1#2}},% Use siunitx for units library
	unit markings={slash space},% The proper way... apparently?
	%
	% Fix micro unit display, see https://tex.stackexchange.com/a/224574/120853:
	x SI prefix/micro/.style={/pgfplots/axis base prefix={axis x base 6 prefix \micro}},
	y SI prefix/micro/.style={/pgfplots/axis base prefix={axis y base 6 prefix \micro}},
	z SI prefix/micro/.style={/pgfplots/axis base prefix={axis z base 6 prefix \micro}},
	%
	% Plot styles:
	regularplot/.style={% A broadly usable, regular plot
			ultra thick,
			axis line style=thick,
			ymajorgrids=true,
			xmajorgrids=true,
			grid style=dashed,
			axis lines=left,
			% scale only axis,% If off, labels are taken into account for size calculations
			ytick align=outside,% Puts ticks outside the plot itself
			xtick align=outside,%
			% Won't work if enlarge limits are invoked before 'axis lines left':
			enlargelimits=0.05,
		},%
	plainplot/.style={%
			% A clean, simple plot when numerical values and labels don't matter
			regularplot,
			xmajorgrids=false,
			ymajorgrids=false,
			ticks=none,% No ticks with numbers
		},
	tuftelike/.style={% https://tex.stackexchange.com/a/155210/120853
			axis line shift=10pt,% Also shifts label automatically
			try min ticks=3,% https://tex.stackexchange.com/a/95753/120853
			max space between ticks=50,% High number so ticks are far apart
			axis lines=left,
			ultra thick,
			axis line style = {semithick, -},% '-' suppresses arrow
			tick style = {semithick, black},%
			ytick align=inside,% Puts ticks inside the plot itself
			xtick align=inside,%
			xtick={%
					% Always set min, max and middle ticks;
					% if more desired, use 'extra x ticks={}'
					\pgfkeysvalueof{/pgfplots/xmin},
					\pgfkeysvalueof{/pgfplots/xmax},
					(\pgfkeysvalueof{/pgfplots/xmax}+\pgfkeysvalueof{/pgfplots/xmin})/2
				},
			ytick={% Same as x
					\pgfkeysvalueof{/pgfplots/ymin},
					\pgfkeysvalueof{/pgfplots/ymax},
					(\pgfkeysvalueof{/pgfplots/ymax}+\pgfkeysvalueof{/pgfplots/ymin})/2
				},
			clip = false,% https://tex.stackexchange.com/a/311194/120853
		},%
	% Unfortunately, decorations lib is relatively verbose:
	arrowplot/.style={
			decoration={
					markings,
					mark=at position 0.25 with {\arrow{#1}},
					mark=at position 0.5 with {\arrow{#1}},
					mark=at position 0.75 with {\arrow{#1}},
				},
			postaction=decorate,
		},
	%
	log x ticks with fixed point/.style={%
			% Logarithmic plot, but with linear labels instead of scientific notation
			% (10^2 becomes 100 etc.)
			% https://tex.stackexchange.com/a/139084/120853
			xticklabel={
					\pgfkeys{/pgf/fpu=true}
					\pgfmathparse{exp(\tick)}%
					\pgfmathprintnumber[fixed relative, precision=3]{\pgfmathresult}
					\pgfkeys{/pgf/fpu=false}
				}
		},
	%
	log y ticks with fixed point/.style={
			yticklabel={
					\pgfkeys{/pgf/fpu=true}
					\pgfmathparse{exp(\tick)}%
					\pgfmathprintnumber[fixed relative, precision=3]{\pgfmathresult}
					\pgfkeys{/pgf/fpu=false}
				}
		}
}

% Required in \tikset body for node placing at x style.
% Requires intersection lib
% https://tex.stackexchange.com/a/93968/120853
\def\parsenode[#1]#2\pgf@nil{%
	\tikzset{label node/.style={#1}}
\def\nodetext{#2}
}

\usepackage{annotate-equations}

% fun highlights
\usepackage[most]{tcolorbox}
\tcbuselibrary{breakable,theorems}
\newtcbox{\hil}[1][razz]{
	on line,
	arc=0pt,outer arc=0pt,
	colback=#1!25!white,
	boxsep=0pt,left=1pt,right=1pt,top=2pt,bottom=2pt,
	boxrule=0pt,
	breakable,enhanced jigsaw,
}

% make sure align*, aligned environments can stretch across pages
\allowdisplaybreaks

% numbering for equations, figures and tables etc
\numberwithin{equation}{section}
\numberwithin{table}{section}
\numberwithin{figure}{section}
\newtheorem{problem}{Problem}[section]
