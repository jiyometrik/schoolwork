\setlength{\headheight}{14.49998pt} % make latex stop complaining!

% simpler commands
\renewcommand{\bf}[1]{\textbf{#1}}
\renewcommand{\it}[1]{\textit{#1}}
\renewcommand{\tt}[1]{\texttt{#1}}
\newcommand{\term}[1]{{\color{razz}\scshape#1}}
\newcommand{\code}[1]{{\color{razz}\ttfamily#1}}

% math macros
\newcommand{\mv}[1]{\ensuremath{\symbf{#1}}}

\newcommand{\f}[2]{\frac{#1}{#2}}
\newcommand{\df}[2]{\dfrac{#1}{#2}}
\newcommand{\lf}[2]{\slashfrac[auto]{#1}{#2}}
\newcommand{\reals}{\symbb{R}}
\newcommand{\rationals}{\symbb{Q}}
\newcommand{\ints}{\symbb{Z}}

\newcommand{\ii}{\mv{\hat{i}}}
\newcommand{\jj}{\mv{\hat{j}}}
\newcommand{\kk}{\mv{\hat{k}}} % if ever needed

% define colours used throughout the document
% \definecolor{accent}{HTML}{EF476F}
\definecolor{razz}{HTML}{DB3069} % pink -- all of these colours are very vibrant
% \definecolor{cobalt}{HTML}{1446A0} % blue
% \definecolor{naples}{HTML}{F5D547} % yellow

% wrap figures
\usepackage{wrapfig}

% maths
% \usepackage{mathtools, amsthm, amssymb, amsmath}
\usepackage{derivative}

% expandable brackets
\usepackage{physics2}
\usephysicsmodule[tightbraces=true]{ab}

% si units
\usepackage{siunitx}
\sisetup{mode=match}

% font
\usepackage{fontspec}
\defaultfontfeatures{Scale=MatchUppercase, Ligatures=TeX}
\usepackage[
  math-style=TeX,
  bold-style=upright,
  sans-style=upright,
  warnings-off={mathtools-colon,mathtools-overbracket}
]{unicode-math}
\setmainfont{libertinus serif}
\setmathfont{libertinus math}
\setsansfont{libertinus sans}
\setmonofont{source code pro}

% fun highlights
\usepackage[most]{tcolorbox}
\tcbuselibrary{breakable,theorems}
\newtcbox{\hil}[1][razz]{
  on line,
  arc=0pt,
  outer arc=0pt,
  colback=#1!25!white,%colframe=#1,
  boxsep=0pt,left=1pt,right=1pt,top=2pt,bottom=2pt,
  boxrule=0pt,%bottomrule=1pt,toprule=1pt,
  breakable,enhanced jigsaw,
}

% make sure align*, aligned environments can stretch across pages
\allowdisplaybreaks

% better chapter headings
% \titleformat{\subsubsection}[hang]{
%   \sffamily\itshape\normalsize\color{accent}
% }{\thesubsubsection}{.5em}{}

% numbering for equations, figures and tables etc
\numberwithin{equation}{section}
\numberwithin{table}{section}
\numberwithin{figure}{section}
