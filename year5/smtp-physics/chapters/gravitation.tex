\chapter{Gravitation}
\label{ch:gravitation}

\section{Jupiter's moons}
\begin{problem}
The planet Jupiter has a number of moons. Data of some of these moons are given.
Plot a graph of \(\log_{10} T\) against \(\log_{10} r\).
\end{problem}

\begin{table}[htpb]
	\footnotesize
	\sisetup{
		table-number-alignment = center,
		round-mode=figures,
		round-precision=3,
		table-auto-round=true,
	}
	\begin{tabular}{l c c c c}
		\toprule
		moon     & period \(T\) / days & mean distance from centre of Jupiter \(r\) / \unit{\giga\metre} & \(\log_{10} T\)  & \(\log_{10} r\)   \\
		\midrule
		Sinope   & \num{758}           & \num{23.7}                                                      & \num{2.88}       & \num{10.37}       \\
		Leda     & \num{239}           & \num{11.1}                                                      & \hil{\num{2.38}} & \hil{\num{10.05}} \\
		Callisto & \num{16.7}          & \num{1.88}                                                      & \hil{\num{1.22}} & \hil{\num{9.27}}  \\
		Io       & \num{1.77}          & \num{.422}                                                      & \hil{\num{.248}} & \hil{\num{8.63}}  \\
		Metis    & \num{.295}          & \num{.128}                                                      & \num{-.53}       & \num{8.11}        \\
		\bottomrule
	\end{tabular}
	\label{tab:jupitermoon}
	\caption{Data on Jupiter's moons (selection)}
\end{table}

The graph is shown in \cref{fig:moons}.

\begin{figure}[htpb]
	\centering
	\begin{tikzpicture}
		\begin{axis}[
				tuftelike,
				title={Graph of \(\log_{10} T\) against \(\log_{10} r\)},
				xlabel={\(\log_{10} r\)},
				ylabel={\(\log_{10} T\)},
				% grid style=dashed,
			]
			\addplot coordinates {(8.11,-.53)(8.63,.248)(9.27,1.22)(10.05,2.38)(10.37,2.88)};
		\end{axis}
	\end{tikzpicture}
	\label{fig:moons}
	\caption{Graph of \(\log_{10} T\) against \(\log_{10} r\)}
\end{figure}

\begin{problem}
Use the data in \cref{tab:jupitermoon} to determine Jupiter's mass.
\end{problem}

Let the mass of Jupiter be \(M\).

Using a graphing calculator, the equation of the line in \cref{fig:moons} was found to be
\begin{equation}
	\label{eq:linreg}
	\lg T = \num{1.47678}\cdot\lg r - \num{12.49742}
\end{equation}
The period of each of Jupiter's moons is controlled by the centripetal
force they experience keeping them in orbit. This centripetal force is provided
by the gravitational force of attraction between the moon and Jupiter. We can write
\begin{align*}
	\f{GM\cancel{m}}{r^2} & = \cancel{m}r\omega^2 = \f{4\pi^2\cancel{m}r}{T^2} \\
	T^2                   & = \f{4\pi^2}{GM} r^3
\end{align*}
We can linearise this:
\begin{equation}
	\label{eq:linearisation}
	\lg T = \f{1}{2}\lg{\f{4\pi^2}{GM}} + \f{3}{2} \lg r
\end{equation}
So we obtain
\begin{align*}
	\f{1}{2}\lg{\f{4\pi^2}{GM}} & = -\num{12.49742}             \\
	M                           & = \hil{\qty{17.9}{\kilogram}}
\end{align*}

\section{Binary stars}
\begin{problem}
In a binary star system, two stars \(A\) and \(B\) follow
circular orbits of radii \(R\) and \(r\) respectively about their
common centre of mass \(O\). The masses of stars \(A\) and \(B\) are \(M\) and \(m\) respectively.
Show that the period of rotation of star \(A\) is equal to that of star \(B\).
\end{problem}
The \hil{gravitational force} acting on \(B\) due to \(A\) provides
the centripetal acceleration required for \(B\) to follow its circular orbit; this holds with
respect to star \(A\) also.
Since the magnitude of this gravitational force \(F_g = \f{GMm}{\ab(R+r)^2}\) is equal
for both stars, so too is the magnitude of the centripetal force (\(\lf{4\pi^2 MR}{{T_A}^2}\) for \(A\), \(\lf{4\pi^2 mr}{T_B^2}\) for \(B\))
acting on either star.

Therefore
\begin{equation}
	\label{eq:equalperiod}
	\f{\cancel{4\pi^2}MR}{T_A^2} = \f{\cancel{4\pi^2}mr}{T_B^2}
\end{equation}

We knew from before that the centre of mass \(O\) is situated between \(A\) and \(B\).
By treating \(O\) as a pivot to a massless rod on which \(A\) and \(B\) balance, we can use the \term{principle of moments}
and say that \(mr = MR\): the anticlockwise moment contributed by \(M\) is equal to the clockwise moment contributed by \(m\).
\begin{equation}
	\label{eq:com}
	\f{R}{r} = \f{m}{M}
\end{equation}
Substituting \cref{eq:com} back into \cref{eq:equalperiod}, we get that \(T_A^2 = T_B^2\quad\Rightarrow T_A = T_B\quad\blacksquare\).

\begin{problem}
Show that the period \(T\) of rotation of the stars is given by
\begin{equation*}
	T^2 = \f{4\pi^2 \ab(R+r)^3}{G\ab(M+m)}
\end{equation*}
\end{problem}
With respect to \(A\), the gravitational force exerted on it by \(B\) provides its centripetal acceleration.
\begin{equation}
	\label{eq:fgfca}
	\f{G\cancel{M}m}{\ab(R+r)^2}=\f{4\pi^2 R\cancel{M}}{T^2}
\end{equation}
This also happens for \(B\).
\begin{equation}
	\label{eq:fgfcb}
	\f{GM\cancel{m}}{\ab(R+r)^2}=\f{4\pi^2 r\cancel{m}}{T^2}
\end{equation}
Adding \cref{eq:fgfca,eq:fgfcb}, we obtain
\begin{equation*}
	\f{G}{\ab(R+r)^2}\cdot\ab(M+m) = \f{4\pi^2}{T^2} \cdot \ab(R+r)
\end{equation*}
This leads to our \textsc{glorious} conclusion:
\begin{equation}
	\label{eq:proven}
	T^2 = \f{4\pi^2 \ab(R+r)^3}{G\ab(M+m)}\quad\blacksquare
\end{equation}
\begin{problem}
For a given binary star system, observations give the period as \(T = \qty{3.42e5}{\second}\)
and the magnitude of the velocity of one star relative to the other as \(\Delta v = \qty{2.26e5}{\metre\per\second}\).
Calculate the total mass of the binary star system.
\end{problem}
We first dissect the meaning of \hil{relative velocity}. With stars \(A\) and \(B\) to illustrate,
\begin{equation*}
	\Delta\mv{v} = \mv{v}_A - \mv{v}_B
\end{equation*}
But since \(A\) and \(B\) have tangential velocities in opposite directions,
\begin{equation*}
	\Delta {v} = v_A - \ab(-v_B) = v_A + v_B
\end{equation*}
For circular motion, \(v = r\omega = \lf{2\pi r}{T}\).
Since \(T\) is identical for both stars,
\begin{equation}
	\label{eq:relv}
	\Delta {v} = v_A + v_B = \ab(R+r)\cdot\f{2\pi}{T}
\end{equation}
We can substitute \cref{eq:relv} into \cref{eq:proven} from previously:
\begin{align*}
	m+M & = \f{\ab(R+r)^3\omega^2}{G}                   \\
	    & = \f{\omega^2}{G}\ab(\f{\Delta{v}}{\omega})^3 \\
	    & = \f{\ab(\Delta{v})^3}{G\omega}               \\
	    & = \f{T\ab(\Delta{v})^3}{2\pi G}               \\
	    & = \hil{\qty{9.42e30}{\kg}}
\end{align*}

\section{Oily planet}
\begin{problem}
A planet consists of a solid core of radius \(R\) covered uniformly with a thick
layer of fluid of thickness \(\lf{R}{2}\). The density of the fluid is \(\rho\) and
that of the solid core is \(2\rho\). Write down the gravitational field strength at \(P\)
and \(Q\) respectively in terms of \(G\), \(R\) and \(\rho\). Sketch the variation of \(g\) with the distance \(r\) from the planet's centre.
\end{problem}
Recall that \(m = \rho V\) for any object of mass \(m\), volume \(V\) and density \(\rho\).

We know that the mass of the spherical \it{solid} region is
\begin{equation*}
	m_s = \f{4}{3}\pi R^3 \cdot 2\rho = \f{8}{3}\pi\rho R^3
\end{equation*}
and the mass of the \it{liquid} region is
\begin{equation*}
	m_l = \f{4}{3} \pi\rho\ab[\ab(\lf{3R}{2})^3 - R^3] = \f{19}{6}\pi\rho R^3
\end{equation*}
Knowing that the gravitational field strength \(g\) a distance of \(r\) away from a point mass \(m\) is given by \(g = \lf{Gm}{r^2}\),
\begin{align}
	\label{eq:gfp}
	g_P & = \f{\lf{8\pi\rho R^3}{3}}{R^2} = \lf{8G\pi\rho R}{3}                  \\
	\label{eq:gfq}
	g_Q & = \f{\lf{31\pi\rho R^3}{6}}{\ab(\lf{3R}{2})^2} = \lf{62G\pi\rho R}{27}
\end{align}
Outside of the planet, we know that \(g \propto \lf{1}{r^2}\). The graph is as follows:
\begin{figure}[htpb]
	\begin{tikzpicture}[
			declare function={
					gwrtr(\x) = (\x<=10) * ((1.161)*\x) + and(\x>10, \x<=15) * (\x) + (\x>15) * ((0.2)/(\x*\x));
				}
		]
		\begin{axis}[
				plainplot,
				title={Graph of \(g\) against \(r\)},
				xlabel={\(r\)},ylabel={\(g\)},
				xmin=0,xmax=25,
				%ymin=0,%ymax=100,
			]
			\pgfplotsinvokeforeach{10, 15}{
				\draw[dashed] ({rel axis cs: 0,0} -| {axis cs: #1, 0}) -- ({rel axis cs: 0,1} -| {axis cs: #1, 0});
			}
			\addplot[smooth] {gwrtr(x)};
		\end{axis}
	\end{tikzpicture}
\end{figure}


\section{Spring balances and the Earth}
\begin{problem}
The Earth may be considered to be a uniform sphere of radius \(R = \qty{6370}{\km}\),
spinning on its axis with a period of \(T = \qty{24.0}{\hour}\). The gravitational field
at the Earth's surface is identical with that of a point mass of \(M = \qty{5.98e24}{\kilogram}\)
at the centre of the Earth. For a \(m = \qty{1.00}{\kilogram}\) mass at the Equator,
\begin{itemize}
	\item calculate the gravitational force acting on the mass,
	\item determine the force required to maintainthe circular path of the mass,
	\item and deduce the reading on an accurate spring balance supporting the mass.
\end{itemize}
\end{problem}
Using \term{Newton's law of gravitation}, we find that
\begin{equation}
	\label{eq:massspringbalance}
	F_g = \f{GMm}{R^2} = \hil{\qty{9.83}{\newton}}
\end{equation}
The force required to maintain the circular path of the mass is the \hil{centripetal force} acting on it.
\begin{equation}
	\label{eq:masscentripetal}
	F_c = mR\omega^2 = \f{4\pi^2mR}{T^2} = \hil{\qty{3.37e-2}{\newton}}
\end{equation}
The gravitational force acting on the mass and the force exerted by the spring balance on it
provide for the centripetal acceleration. Therefore \(F_c = F_g - N\), and \(N = \hil{\qty{9.80}{\newton}}\) from \cref{eq:massspringbalance,eq:masscentripetal}.

\begin{problem}
What would the acceleration of the mass on the Earth's surface due to
\begin{itemize}
	\item the gravitational force alone?
	\item the force measured on the spring balance?
\end{itemize}
\end{problem}
\hil{\qty{9.83}{\meter\per\second\squared}} and \hil{\qty{9.80}{\meter\per\second\squared}} respectively.
\begin{problem}
Suppose a mass \(m\) is hung on the spring balance, and the system is now
brought to a latitude of angle \(\theta\). Deduce a general expression for the force measured
by the spring balance in terms of \(\theta\), and define all other symbols used.
\end{problem}
Let \(G\) be the gravitational constant, \(M\) be the mass of the Earth,
\(R\) be the Earth's radius, and \(T\) be its period of rotation (a day).

At the latitudes, the gravitational force acting on the mass and the force exerted by the spring balance on the mass still provides its centripetal acceleration.
Since the spring balance is vertical to the mass, we only consider the vertical component of the normal force \(N \csc \theta\).
\begin{equation*}
	\f{GMm}{R^2} - N\csc\theta = mR\omega^2 = \f{4\pi^2mR}{T^2}
\end{equation*}
From here,
\begin{equation}
	N = \ab(\f{GM}{R^2} - \f{4R\pi^2}{T^2})m\sin\theta
\end{equation}