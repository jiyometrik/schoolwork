\chapter{Energy}

\section{Bead and wire}
\begin{problem}
  Under the influence of gravity, a bead slides along a frictionless wire whose
  height is given by the function \(y\ab(x)\). At position \(\ab(x_0,
  y_0) = \ab(0,0)\),
  the wire is vertical. The bead passes this point with a speed
  \(V\) downwards.

  What should the shape of the wire be such that the vertical speed
  remains \(V\) at all times?
\end{problem}

Treat \(x\) and \(y\) as functions of time \(t\), that is,
\(x\ab(t)\) and \(y\ab(t)\). The horizontal and vertical speeds can
be written as \(v_x = \odv{x\ab(t)}{t}\) and \(v_y = \odv{y\ab(t)}{t}\).

We have the condition in \cref{eq:vertspeed}.
\begin{equation}
  \odv{y\ab(t)}{t} = V\qquad y(0) = 0
  \label{eq:vertspeed}
\end{equation}
Integrating \cref{eq:vertspeed} w.r.t. \(t\), we obtain
\cref{eq:vertdistance}, which satisfies the initial condition.
\begin{equation}
  y\ab(t) = Vt + y(0) = Vt
  \label{eq:vertdistance}
\end{equation}

According to the \term{law of conservation of energy}, the ball's
energy must be conserved at all points of the wire. We can write the
ball's total energy as the sum of its gravitational potential energy
and kinetic energy at a time \(t\).
\begin{equation*}
  E\ab(t) = mgy\ab(t) + \f{1}{2}m\ab[v\ab(t)]^2
\end{equation*}
And since \(v\ab(t) = \sqrt{{v_x}^2 + {v_y}^2}\), we can say that
\begin{equation}
  E\ab(t) = mgy\ab(t) + \f{1}{2}m\ab[\ab(\odv{x\ab(t)}{t})^2 +
  \ab(\odv{y\ab(t)}{t})^2]
\end{equation}
At \(t = 0\), \(y\ab(t) = 0\). As such, the ball's initial energy is
\begin{align*}
  E\ab(0) = \cancelto[razz]{0}{mgy\ab(0)} +
  \f{1}{2}m\ab[\ab(\odv{x\ab(t)}{t})^2 +
  \cancelto[razz]{V}{\ab(\odv{y\ab(t)}{t})}^2]
\end{align*}
