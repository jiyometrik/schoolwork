\chapter{Energy}

\section{String and pucks}

This problem is taken from David Morin's {\scshape Introduction to
Classical Mechanics: With Problems and Solutions}
as \bf{Problem 4.11.4} (`Pulling the pucks').

\begin{problem}
  A string of length \(2l\) connects two hockey pucks that lie on
  frictionless ice (\cref{fig:pucks}). A constant horizontal force
  \(\mv{F}\) is applied
  to the midpoint of the string, perpendicular to it. How much
  kinetic energy is lost when the pucks collide, assuming they stick together?
\end{problem}

\begin{wrapfigure}{o}{0.5\textwidth}
  \centering
  \begin{tikzpicture}
    \filldraw[razz] (0,0) circle(2pt);
    \filldraw[razz] (0,3) circle(2pt);
    \draw(0,0)--(0,3);
    \draw[-Stealth,ultra thick] (0,1.5)--(2,1.5);
    \node at (1.2,1.8){$\mv{F}$};
  \end{tikzpicture}
  \caption{String and two pucks}
  \label{fig:pucks}
\end{wrapfigure}

Let \(\theta\) be the angle between the stretched string (along which
tension \(\mv{T}\) also lies) and \(\mv{F}\). The tension in the
string \(\mv{T} = \lf{\mv{F}}{2\cos\theta}\). Looking at the top
puck, the component of the tension in the \(y\)--direction is
\(-\mv{T}\sin\theta=-\lf{\mv{F}\tan\theta}{2}\). Since the force
applied is the displacement derivative of the work done,
\begin{wrapfigure}{o}{0.5\textwidth}
  \centering
  \begin{tikzpicture}
    \filldraw[razz] (0,0) circle(2pt);
    \filldraw[razz] (0,3) circle(2pt);
    \draw(0,0)--(1,1.5)--(0,3);
    \draw[-Stealth,ultra thick, dashed] (0,1.5)--(2,1.5);
    \node at (1.2,1.8){$\mv{F}$};
    \node at (.9,1.52){$\theta$};
    \node at (0.5,.77){$\mv{T}$};
  \end{tikzpicture}
  \caption{String and two pucks, stretched}
  \label{fig:pucksstretched}
\end{wrapfigure}
\begin{align*}
  W_y &= \int_{l}^{0}-\f{F\tan\theta}{2}\:\odif{y} \\
  &= \int_{\lf{\pi}{2}}^{0}-\f{F\tan\theta}{2}\:\odif{\ab(l\sin\theta)}\\
  &= \int_{\lf{\pi}{2}}^{0} -\f{Fl\sin\theta}{2}\:\odif{\theta}\\
  &= \ab[\f{Fl\cos\theta}{2}]_{\lf{\pi}{2}}^0 \\
  &= \f{1}{2}Fl
\end{align*}
By the \term{work-energy theorem}, this work equals \(W_y =
\lf{mv_y^2}{2}\). The kinetic energy lost when the two pucks stick
together is twice this quantity, since \(v_x\) doesn't change.
Therefore the loss in kinetic energy is \hil{\(Fl\)}.

\section{Bead and wire}
This problem is taken from David Morin's {\scshape Introduction to
Classical Mechanics: With Problems and Solutions}
as \bf{Problem 4.11.6} (`Constant \(\dot{y}\)').

\begin{problem}
  Under the influence of gravity, a bead slides along a frictionless
  wire whose height is given by the function \(y\ab(x)\). At position
  \(\ab(x_0, y_0) = \ab(0,0)\), the wire is vertical. The bead passes
  this point with a speed \(V\) downwards.

  What should the shape of the wire be such that the vertical speed
  remains \(V\) at all times?
\end{problem}

Treat \(x\) and \(y\) as functions of time \(t\), that is,
\(x\ab(t)\) and \(y\ab(t)\). The horizontal and vertical speeds can
be written as \(v_x = \dot{x}\) and \(v_y = \dot{y}\).

We have the condition in \cref{eq:vertspeed}.
\begin{equation}
  \dot{y} = V\qquad y\ab(0) = 0
  \label{eq:vertspeed}
\end{equation}
Integrating \cref{eq:vertspeed} w.r.t. \(t\), we obtain
\cref{eq:vertdistance}, which satisfies the initial condition.
\begin{equation}
  y\ab(t) = Vt + y\ab(0) = Vt
  \label{eq:vertdistance}
\end{equation}

According to the \term{law of conservation of energy}, the ball's
energy must be conserved at all points of the wire. We can write the
ball's total energy as the sum of its gravitational potential energy
and kinetic energy at a time \(t\).
\begin{equation*}
  E\ab(t) = mgy\ab(t) + \f{1}{2}m\ab[v\ab(t)]^2
\end{equation*}
And since \(v\ab(t) = \sqrt{{v_x}^2 + {v_y}^2}\), we can say that
\begin{equation}
  E\ab(t) = mgy\ab(t) + \f{1}{2}m\ab(\dot{x}^2 + \dot{y}^2)
\end{equation}
At \(t = 0\), \(x = 0\) and \(y = 0\). As such, the ball's initial energy is
\begin{align*}
  E\ab(0) = \cancelto{0}{mgy\ab(0)} +
  \f{1}{2}m\ab(\cancelto{0}{\dot{x}^2} + V^2)
  = \f{mV^2}{2}
\end{align*}

Using the law of \term{conservation of energy}, \(E\ab(0) = E\ab(t)\)
for any time \(t\).
\begin{align*}
  E\ab(0) &= E\ab(t)\\
  \f{\cancel{m}V^2}{2} &= \cancel{m}gy + \f{1}{2}\cancel{m}\ab(\dot{x}^2 + V^2)
\end{align*}
\begin{equation}
  \label{eq:horzspeed}
  v_x = \dot{x} = \pm\sqrt{-2gy}
\end{equation}

Using the \term{chain rule}, \(\odv{y}/{x} = \lf{\dot{y}}{\dot{x}}\).
With \cref{eq:vertspeed,eq:horzspeed}, we get an expression for \(\odv{y}/{x}\):
\begin{equation}
  \odv{y}{x} = \pm\f{V}{\sqrt{-2gy}}
\end{equation}
Integrating this is simple.
\begin{align*}
  \odv{y}{x} &= \pm\f{V}{\sqrt{-2gy}} \\
  \sqrt{-2g}\int\sqrt{y}\:\odif{y} &= \pm\int V\:\odif{x}  \\
  \f{2\sqrt{-2g}}{3}y^{\lf{3}{2}} + \cancelto{0}{y\ab(0)} &= \pm Vx +
  \cancelto{0}{x\ab(0)}
\end{align*}

And we get an expression for \(y\) in terms of \(x\).\sidenote{Since
the wire always points downwards, we only consider the negative option.}
\begin{equation}
  y = -\ab(Vax)^{\lf{2}{3}}\qquad a = \f{3}{2\sqrt{-2g}}
\end{equation}
