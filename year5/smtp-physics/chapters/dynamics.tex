\chapter{Dynamics}

\section{Two pulleys}
\begin{figure}[htpb]
  \centering
  \includegraphics{assets/twopulleys.png}
  \caption{A two-pulley system}
  \label{fig:twopulleys}
\end{figure}

In \cref{fig:twopulleys}, a mass \(m_1\) hangs from a massless string
that passes over a very light fixed pulley \(P_1\).
The same string is connected to a second light pulley \(P_2\). A
second string passes around \(P_2\) with one end
attached to a wall and the other to another mass \(m_2\) on a
frictionless, horizontal table.

\it{If \(a_1\) and \(a_2\) are the accelerations of \(m_1\) and
  \(m_2\) respectively,
  deduce the tensions \(T_1\) and \(T_2\) in \(m_1\) and \(m_2\)
  (respectively) in terms of
  \(m_1\), \(m_2\) and \(g\).
}

Taking downwards as \(+y\), and by Newton's Third Law,
\begin{align}
  \label{eq:m1a1}
  m_1a_1 &= m_1g-T_1 \\
  m_2a_2 &= T_2
  \label{eq:m2a2}
\end{align}
We observe also that \(P_2\) is being pulled on by \(P_1\) and
\(m_2\). Since its
mass is negligible, the net force acting on it must be zero.
\begin{equation}
  0 = 2T_2 - T_1
  \label{eq:t1t2}
\end{equation}
We also observe that \(P_2\) coils twice for every coil \(P_1\) makes, so
\begin{equation}
  a_1 = 2a_2
  \label{eq:a1a2}
\end{equation}

Substituting \cref{eq:t1t2} and \cref{eq:a1a2} into \cref{eq:m1a1},
\begin{align*}
  m_1g - 2T_2 &= m_1a_1 \\
  &= 2m_1a_2 \\
  &= \f{2m_1T_2}{m_2} \\
  \therefore\; T_2\ab(\f{2m_1}{m_2}+2) &= m_1g \\
  T_2 &= \f{m_1g}{ \f{2\ab(m_1+m_2)}{m_2} } \\
  % &= \hil{ \f{m_1m_2g}{2\ab(m_1+m_2)} }
  &= \f{m_1m_2g}{2\ab(m_1+m_2)}
\end{align*}

Recalling \cref{eq:t1t2},
\begin{align*}
  T_1 &= 2T_2 \\
  &= \f{m_1m_2g}{m_1+m_2}
\end{align*}

\section{See-saw}
Two people are standing on a \qty{2.0}{\metre} long platform, one at
each end. The platform floats parallel
to the ground on a cushion of air, like a hovercraft. One person
throws a \qty{6.0}{\kg} ball
to the other, who catches it. The ball travels nearly horizontally.
Excluding the ball, the total mass of the platform and people is \qty{120}{\kg}.
Because of the throw, this \qty{120}{\kg} mass recoils. \it{How far
does it move before coming to rest again?}

Assuming the platform to be uniform and both people to be of the same
mass, we define
\begin{itemize}
  \item \(m_b\) and \(m_p\) to be the masses of the ball and the
    people and platform combined, respectively.
  \item \(\mv{s}_b\) and \(\mv{s}_p\) to be the positions of the ball
    and the platform (with people).
  \item \(x\) to be the centre of mass of the system\sidenote{My
      solution uses ideas about \it{centre of mass}; there should be a
    solution using ideas about momentum, however.}.
\end{itemize}

Taking the centre of the platform to be the origin,
we also know that the \term{centre of mass} of the platform is
\begin{align*}
  x &= \f{m_b \mv{s}_b + m_p \mv{s}_p}{m_b + m_p} \\
  &= \f{6.0 \times 1.0 + 120 \times 0.0}{120 + 6.0} \\
  &= \f{6.0}{126}
\end{align*}

Since no external force acts on the system, the centre of mass remains the same.
The ball has shifted from \(\mv{s}_b = 0\) to \(\mv{s}_b' =
\qty{-1.0}{\metre}\), so
\begin{align*}
  x' &= \f{m_b \mv{s}_b' + m_p \mv{s}_p'}{m_b + m_p} \\
  &= \f{6.0\ab(-1.0) + 120\Delta\mv{s}}{120 + 6.0} \\
  120\Delta\mv{s} - 6.0 &= 6.0 \\
  \Delta\mv{s} &= \hil{\qty{+0.10}{\metre}}
\end{align*}

\section{Two carts}
Figure blabla shows the momentum--time graph for a collision between
two carts with masses of \qty{1.0}{\kg}
and \qty{3.0}{\kg} respectively.

\it{Show that the collision of the carts obey the principle of
conservation of linear momentum.}
The principle of conservation of linear momentum states that
\(\Delta\mv{p}_1 = -\Delta\mv{p}_2\).
\begin{align*}
  \Delta\mv{p}_1 &= 0.0 - 2.0 \\
  &= \qty{-2.0}{\kg\m\per\second} \\
  \Delta\mv{p}_2 &= 1.0 - \ab(-1.0) \\
  &= \qty{2.0}{\kg\m\per\second} \\
  \sum\Delta\mv{p} &= -2.0 + 2.0 \\
  &= \hil{\qty{0}{\kg\m\per\second}}
\end{align*}

\it{Show that the force acting on cart 1 by cart 2 is equal to that
acting on cart 2 by cart 1 during collision.}
According to the graph, the time taken for the collision is the same
for cart 1 on cart 2
and vice versa. Since
\begin{align*}
  \Delta\mv{p}_1 &= -\Delta\mv{p}_2 \\
  \Rightarrow \mv{J}_1 &= -\mv{J}_2 \\
  \odv{\mv{J}_1}{t} &= -\odv{\mv{J}_2}{t} \\
  \therefore\; \mv{F}_1 &= -\mv{F}_2
\end{align*}

\it{Explain if this collision is an elastic, inelastic or perfectly
inelastic collision.}
The \term{elasticity} of a collision is dependent on the loss of kinetic energy
from the system after the collision.
\begin{align*}
  \text{KE} &= \f{1}{2}m\mv{v}^2 \\
  &= \f{\mv{p}^2}{2m} \\
  \sum \text{KE} &= \f{\ab(2.0)^2}{2\times 1.0} +
  \f{\ab(-1.0)^2}{2\times 3.0} \\
  &= \qty{2.2}{\joule} \\
  \sum \text{KE}' &= \f{\ab(0.0)^2}{2\times 1.0} +
  \f{\ab(1.0)^2}{2\times 3.0} \\
  &= \qty{0.2}{\joule}
\end{align*}
Since \(\Delta\text{KE} < 0\), some (but not all) kinetic energy was
lost from the system. The collision is \hil{inelastic}.

\section{Putty lump}
\begin{wrapfigure}{o}{0.4\textwidth}
  \centering
  \includegraphics[scale=0.5]{assets/puttylump.png}
  \caption{Suspended frame containing putty lump}
  \label{fig:puttylump}
\end{wrapfigure}
In \cref{fig:puttylump}, a \qty{0.150}{\kg} frame, when suspended from a coil
spring, stretches the spring \qty{0.0400}{\metre}. A \qty{0.200}{\kg} lump of
putty is dropped from rest onto the frame from a height of \qty{30.0}{\cm}.
\it{Find the maximum distance the frame moves downward from its
initial equilibrium position.}

We can first find the spring constant, since the weight of the frame
provides the elastic force in the spring.
\begin{align*}
  m_f\mv{g} &= k\mv{x} \\
  k &= \lf{m_f\mv{g}}{\mv{x}} \\
  &= \lf{0.150\times 9.81}{0.070} \\
  &= \qty{21.02}{\newton\per\metre}
\end{align*}

Taking downwards as \(+y\), the terminal velocity of the putty is
\begin{align*}
  \mv{u}_p &= \sqrt{2g\mv{s}_p} \\
  &= \sqrt{2\ab(-9.81)\ab(-0.300)} \\
  &= \qty{2.426}{\metre\per\second}
\end{align*}
The collision between the lump of putty and frame is perfectly
inelastic because the putty is assumed to
have adhered to the frame. By the \term{principle of conservation of momentum},
\begin{align*}
  m_p\mv{u}_p + 0 &= m_p\mv{v}_p + m_f\mv{v}_f \\
  \mv{v}_f &= \f{m_p\mv{u}_p}{m_p + m_f}
\end{align*}

By the \term{law of conservation of energy}, all kinetic energy will be
converted to elastic potential energy at the instant where the
maximum stretch distance is reached.
\begin{align*}
  \frac{1}{2}m_f{\mv{v}_f}^2 &= \frac{1}{2}k{\mv{x}'}^2 \\
  \mv{x}' &= \sqrt{\f{m_f\ab(\f{m_p\mv{u}_p}{m_p + m_f})^2}{k}} \\
  &= \sqrt{\f{0.150\times \ab(\f{0.200\times 2.426}{0.150 +
  0.200})^2}{21.02}} \\
  &= \qty{0.117}{\metre} \\
  \Delta\mv{s} &= \mv{x}' - 0.0400 \\
  &= \hil{\qty{0.077}{\metre}}
\end{align*}

\section{Rocket fuel}
A rocket of empty mass \(M_0\) carries fuel with an effective mass \(m\).
The fuel is combusted and ejected out of the rocket at a rate of
\(\odv{m}{t} = \lambda\)
with speed \(v_e\) relative to the rocket. \it{Derive the equation of
motion of a rocket travelling at speed \(v\) inside Earth.}\sidenote{Unsolved.}

At any time \(t\), the mass of the rocket \(M\ab(t)\) is
\begin{equation}
  M\ab(t) = M_0 + m\ab(t)
  \label{eq:mt}
\end{equation}

\section{Falling sand}
Sand falls through a hole in a sandbag at a rate of
\qty{0.100}{\kg\per\second} and lands
on the pan of a weighing machine directly below. The hole is
\qty{1.00}{\metre} above the pan.

\it{Find the velocity of the sand just before it hits the pan.}
Neglecting air resistance, the sand is in \term{free fall}. The
\term{terminal velocity} is thus
\(v = \sqrt{2gh}\), where \(g\) is the gravitational constant and
\(h\) is the height from which
the sand has fallen.
\begin{align*}
  v &= \sqrt{2 g h} \\
  &= \sqrt{2 \times \ab(-9.81) \times \ab(-1.00)} \\
  &= \hil{\qty{4.43}{\metre\per\second}}
\end{align*}

\it{Assuming the sand stops when it hits the pan, what is the change
of momentum of the sand per second due to the impact?}
We know that \(\Delta\mv{p} = m\Delta\mv{v}\).
\begin{align*}
  \Delta\mv{p} \text{ (per second)} &= m\Delta\mv{v} \\
  &= 0.100 \times 4.43 \\
  &= \hil{\qty{0.443}{\kg\metre\per\second}}
\end{align*}

\it{What is the force on the pan due to the impact alone?}
We know that \(\mv{F} = \odv{\mv{p}}{t}\). therefore, \(\mv{F} =
\lf{\Delta\mv{p}}{\Delta t} = \hil{\qty{0.443}{\newton}}\).

\it{If the total mass of the sand is \qty{1.00}{\kg}, what is the
  reading of the balance (in \unit{\kg}) when the last grain of sand
hits the pan?}
The reading on a weighing balance is \(\mv{F}/g\), where \(\mv{F}\)
is the net force exerted on the surface.
\begin{align*}
  \mv{F} &= -\ab[-\mv{g} + \ab(-\mv{F}_\text{sand--pan})] \\
  &= 9.81 + 0.443 \\
  &= \qty{10.253}{\newton}\\
  \text{reading} &= \lf{10.253}{g} \\
  &= \hil{\qty{1.05}{\kg}}
\end{align*}

\section{Colliding atoms}
A neon atom (\(m = \qty{20.0}{\dalton}\)) makes a perfectly elastic
collision with another atom at rest.
After the impact, the neon atom travels away at a \qty{55.6}{\degree}
angle from its original direction
and the unknown atom travels away at a \qty{-50.0}{\degree} angle.
\it{What is the mass of the unknown atom, in \unit{\dalton}?}

Let
\begin{itemize}
  \item \(\mv{u}\) be the initial horizontal velocity of the neon
    atom, assuming that there is always an initial horizontal
    velocity that produces such a glancing collision.
  \item \(m_1\) and \(m_2\) be the masses of the neon atom and
    unknown atom respectively.
  \item \(\alpha\) and \(\beta\) be the angles of deflection of the
    neon atom and unknown atom respectively.
  \item \(\mv{v}_1\) and \(\mv{v}_2\) be the final velocities of the
    neon atom and unknown atom respectively.
\end{itemize}
The goal here is to obtain an expression for \(m_2\), in terms of the
given factors
\(m_1\), \(\alpha\), and \(\beta\).

The \term{law of conservation of momentum} dictates that, for a
head-on collision,
\begin{equation}
  m_1\mv{u}_1 + m_2\mv{u}_2 = m_1\mv{v}_1 + m_2\mv{v}_2
\end{equation}

Applying this in the horizontal and vertical directions respectively,
\begin{align}
  \label{eq:pconvx}
  m_1\mv{u} &= m_1\mv{v}_1 \cos\alpha + m_2\mv{v}_2 \cos \beta \\
  0 &= m_1\mv{v}_1 \sin \alpha - m_2\mv{v}_2 \sin \beta
  \label{eq:pconvy}
\end{align}

Since the collision was \term{elastic}, the relative speed of approach
is equal to the relative speed of separation.
\begin{equation}
  \mv{u} = \mv{v}_2 \cos \beta - \mv{v}_1 \cos \alpha
  \label{eq:approachspeed}
\end{equation}

We can rewrite \cref{eq:pconvy} as
\begin{equation}
  m_2\mv{v}_2 = \dfrac{m_1\mv{v}_1\sin\alpha}{\sin\beta}
  \label{eq:mv2}
\end{equation}

Substituting the result from \cref{eq:mv2} into \cref{eq:pconvx}, we obtain
\begin{align*}
  m_1\mv{u} &= m_1\mv{v}_1\cos\alpha +
  \dfrac{m_1\mv{v_1}\sin\alpha}{\sin\beta}\cos\beta \\
\end{align*}
\begin{equation}
  \mv{u} = \mv{v}_1\ab(\cos\alpha + \sin\alpha\cot\beta)
  \label{eq:uv1}
\end{equation}

Similarly, from \cref{eq:approachspeed}, we can also write
\begin{equation}
  \mv{v}_2 = \dfrac{\mv{u} + \mv{v}_1\cos\alpha}{\cos\beta}
  \label{eq:v2}
\end{equation}

Substituting \cref{eq:v2} into \cref{eq:mv2}, we obtain
\begin{align*}
  m_2 &= \dfrac{m_1\mv{v}_1\sin\alpha}{\sin\beta} \cdot \dfrac{1}{\mv{v}_2} \\
  &=
  \dfrac{m_1\mv{v}_1\sin\alpha}{\sin\beta}\cdot\dfrac{\cos\beta}{\mv{u}
  + \mv{v}_1\cos\alpha} \\
\end{align*}
\begin{equation}
  \therefore\; m_2
  =\dfrac{m_1\mv{v}_1\sin\alpha\cot\beta}{\mv{u}+\mv{v}_1\cos\alpha}
  \label{eq:m2}
\end{equation}

Returning from \cref{eq:uv1} and substituting into \cref{eq:m2}, we finally get
\begin{equation}
  m_2 = \dfrac{m_1\sin\alpha\cot\beta}{2\cos\alpha + \sin\alpha\cot\beta}
\end{equation}

Using this result (after long last!), we obtain:
\begin{align*}
  m_2 &= \dfrac{m_1\sin\alpha\cot\beta}{2\cos\alpha + \sin\alpha\cot\beta} \\
  &=
  \ab|\frac{20\sin\qty{55.6}{\degree}\cot\ab(\qty{-50.0}{\degree})}{2\cos\qty{55.6}{\degree}
  + \sin\qty{55.6}{\degree}\cot\ab(\qty{-50.0}{\degree})}| \\
  &= \hil{\qty{31.6}{\dalton}}
\end{align*}

\section{Rocket explosion}
A rocket is fired vertically upward. At the instant it reaches an
altitude of \qty{1000}{\metre} and a
speed of \qty{300}{\metre\per\second}, it explodes into three
fragments of equal mass. One fragment moves
upward with a speed of \qty{450}{\metre\per\second} following the
explosion, while another has a speed of
\qty{240}{\metre\per\second} eastwards. \it{What is the velocity of the
last fragment immediately after the explosion?}

Throughout this problem, \(+y\) is northward, and \(+x\) eastward.

The final velocity of the rocket, \(\mv{v}_1\), is
\begin{align*}
  \mv{v}_1 &= \sqrt{\mv{v}_0^2 + 2\mv{a}\mv{s}}\\
  &= \sqrt{\ab(300)^2 + 2\ab(-9.81)(1000)} \\
  &= \qty{265.3}{\metre\per\second}
\end{align*}
Let the three fragments be \(A\), \(B\) and \(C\), each with mass
\(m\). By the \term{law of conservation of momentum}\sidenote{\(\ii\)
and \(\jj\) are unit vectors in the \(x\)-- and \(y\)--directions.},
\begin{align*}
  m(\mv{v}_A + \mv{v}_B + \mv{v}_C) &= -3m\mv{v}_1 \\
  \mv{v}_C &= -3\mv{v}_1 - \ab(\mv{v}_A + \mv{v}_B) \\
  &= -3\ab(v_1\jj) - 450\jj - 240\ii \\
  &= -240\ii - (3v_1 + 450)\jj \\
  \ab|\mv{v}_C| &= \sqrt{\ab(-240)^2 + \ab[-(3v_1+450)]^2}\\
  &= \hil{\qty{831.3}{\metre\per\second}} \\
  \text{direction of } \mv{v}_C &= \arctan\f{3v_1 +450}{240} \\
  &= \hil{\qty{73}{\degree}\text{ south of west}}
\end{align*}

\section{Hydrogen molecules}
When two hydrogen atoms of mass \(m\) combine to form a diatomic
hydrogen molecule,
the potential energy of the system after they combine is \(-A\),
where \(A\) is a positive quantity called the
\term{binding energy} of the molecule.

\it{Show that in a collision that involves only two hydrogen atoms,
  it is impossible to form a
  hydrogen molecule because momentum and energy cannot simultaneously
be conserved.}

Let \(\mv{u}_n\) be the initial velocity of the \(n\)--th hydrogen molecule, and
\(\mv{v}_n\) be its final velocity. Let \(\mv{v}\) be the final velocity of the
hydrogen molecule formed.

Initially, the total momentum is \(\mv{p} = m\ab(\mv{u}_1 + \mv{u}_2)
= 0\). Since momentum
must be conserved, \(\mv{p}' = 2m\mv{v} = 0\). It follows that
\(\mv{v} = 0\), so
the hydrogen molecule must be at rest after the collision.
\begin{equation}
  \mv{v} = 0
  \label{eq:finalv}
\end{equation}

Initially, the total energy in the system is \(E = \f{1}{2}m{u_1}^2 +
\f{1}{2}m{u_2}^2 = \f{m\ab({u_1}^2 + {u_2}^2)}{2}\).
After the collision, some energy has been converted to potential
energy (due to the aforementioned binding energy),
hence
\begin{equation}
  E' = \f{2mv^2}{2} - A
  \label{eq:finalE}
\end{equation}

Substituting \cref{eq:finalv} into \cref{eq:finalE}, we get \(E' = -A\).
However, since \(E' > A > 0\), \hil{the above finding is contradictory}.

A hydrogen molecule can be formed in a collision involving three
hydrogen atoms.
Before such a collision, each of the three atoms has a speed of
\(\mv{u}=\qty{1.00e4}{\metre\per\second}\). They approach at angles
of \qty{120}{\degree}
so that at any instant, the atoms lie at the corners of an equilateral triangle.
The binding energy of hydrogen is \(A = \qty{7.23e-19}{\joule}\), and the mass
of a hydrogen atom is \(m = \qty{1.67e-27}{\kg}\). \it{Find}
\begin{itemize}
  \item \it{the speed of the resulting hydrogen molecule;}
  \item \it{the speed of the remaining hydrogen atom.}
\end{itemize}

Let the final speed of the hydrogen molecule and hydrogen atom be
\(\mv{v}\) and \(\mv{w}\) respectively,
and let their directions of travel be \(\alpha\) and \(\beta\).
We can apply the \term{law of conservation of momentum} in both the
\(x\) and \(y\) directions,
taking rightwards and upwards as \(+x\) and \(+y\). Since \(\mv{p} = -\mv{p}'\),
\begin{align}
  \label{eq:hconvx}
  m\ab[\mv{u}\cos\qty{30}{\degree} + (-\mv{u}\cos\qty{30}{\degree}) +
  0] &= -\ab(2m\mv{v}\cos\alpha + m\mv{w}\cos\beta) \\
  \label{eq:hconvy}
  m\ab(2\mv{u}\sin\qty{30}{\degree}-\mv{u}) &=
  -\ab(2m\mv{v}\sin\alpha + m\mv{w}\sin\beta)
\end{align}
Since energy is conserved too, \(E = E'\).
\begin{equation}
  \f{3mu^2}{2} = \f{1}{2}\ab(mv^2 + mw^2) - A
  \label{eq:econv}
\end{equation}
Rearranging \cref{eq:econv}, we get
\begin{equation}
  v^2 + w^2 = \f{3mu^2 + 2A}{m}
  \label{eq:econv2}
\end{equation}
From \cref{eq:hconvx,eq:hconvy} respectively, we get
\begin{align*}
  \mv{v} &= -\f{\mv{w}\cos\beta}{2\cos\alpha} \\
  \mv{v} &= -\f{\mv{w}\sin\beta}{2\sin\alpha} \\
  \therefore\; \lf{\cos\beta}{\cos\alpha} &= \lf{\sin\beta}{\sin\alpha} \\
  \Rightarrow\; \tan\alpha &= \tan\beta
\end{align*}
We conclude that, miraculously,
\begin{equation}
  \alpha = \beta
  \label{eq:alphabeta}
\end{equation}
Substituting \cref{eq:alphabeta} back into \cref{eq:hconvx},
\begin{equation}
  \mv{w} = -2\mv{v}
  \label{eq:w2v}
\end{equation}
Substituting this \term{glorious} discovery \cref{eq:w2v} back into
\ref{eq:econv2},
\begin{align*}
  \mv{v}^2 + 4\mv{v}^2 &= \f{3mu^2+2A}{m} \\
  \mv{v} &= \sqrt{\f{3mu^2+2A}{5m}} \\
  &= \hil{\qty{1.53e4}{\metre\per\second}}\\
  \therefore\; \mv{w} &= -2\sqrt{\f{3mu^2+2A}{5m}}\\
  &= \hil{\qty{-3.05e4}{\metre\per\second}}
\end{align*}
We note that the hydrogen molecule and hydrogen atom travel in
exactly opposite directions, since \(\alpha=\beta\) while the signs
of \(\mv{v}\) and \(\mv{w}\)
are opposite.

\section{Staircase}
A ball bounces inelastically down a flight of steps in a plane
perpendicular to the front edges of the steps.
The horizontal component of the velocity remains constant throughout
but the vertical component of the velocity is reduced to a fraction
\(e\) of the vertical velocity before impact: \(\mv{v}_y' = e\mv{v}_y\).

Each step is \qty{0.20}{\metre} high and \qty{0.30}{\metre} deep.
The ball always bounces exactly in the middle of each step; after
each bounce it rises to the height of the previous step.
Air resistance can be neglected.\sidenote{\(+x\) and \(+y\) are
rightwards and upwards, respectively.}

\it{Show that the value of the constant \(e\) for these impacts is \num{0.71}.}

In the vertical direction, \(\mv{v}_y' = \sqrt{\mv{v}_y^2 +
2\mv{a}\mv{s}}\). With \(\mv{v}_y = 0\)
at the ball's maximum height,
\begin{align*}
  \mv{v}_y' &= \sqrt{2\ab(-g)\ab(-0.20)} \\
  &= \sqrt{0.4g}
\end{align*}

Additionally, for every vertical drop the ball undergoes starting
from the edge of the previous
step to the platform of the next, \(\mv{v}_y' = \sqrt{\mv{v}_y^2 +
2\mv{a}\mv{s}}\) also.
\begin{align*}
  \mv{v}_y &= \sqrt{\mv{v}_y'^2 - 2\mv{a}\mv{s}} \\
  &= \sqrt{0.4g - 2\ab(-g)\ab(0.20)} \\
  &= \sqrt{0.8g}
\end{align*}

Comparing \(\mv{v}_y\) and \(\mv{v}_y'\),
\begin{align*}
  e &= \lf{\mv{v}_y'}{\mv{v}_y} \\
  &= \lf{ \sqrt{0.4g} }{ \sqrt{0.8g} } \\
  &= \lf{1}{ \sqrt{2} } \\
  &= \hil{0.71}
\end{align*}

\it{Show also that the horizontal component of the velocity of the
ball is \qty{0.62}{\metre\per\second}.}

The time taken between collisions is the sum of the time of descent
and time of ascent
of the ball. Since \(\mv{v}' = \mv{v} + \mv{a}t\), the time \(t\)
between collisions is
\begin{align*}
  t &= \lf{\mv{v}_y}{\mv{a}} + \lf{\mv{v}_y'}{\mv{a}} \\
  &= \f{\sqrt{0.4g} + \sqrt{0.8g}}{g} \\
  &= \f{\sqrt{0.4} + \sqrt{0.8}}{\sqrt{g}}
\end{align*}
Since the horizontal velocity \(\mv{v}_x\) is constant throughout,
\(\mv{s}_x = \mv{v}_xt\):
\begin{align*}
  \mv{v}_x &= \lf{\mv{s}_x}{t} \\
  &= \f{\mv{s}_x\sqrt{g}}{\sqrt{0.4} + \sqrt{0.8}} \\
  &= \hil{\qty{0.62}{\metre\per\second}}
\end{align*}

At the bottom of the steps, the ball continues to bounce along a
horizontal pavement.
The collisions of the ball with the pavement have the same value of
\(e = 0.71\).
\it{Show that the ball does not bounce beyond a distance from the
bottom of the steps, and find this distance.}

After the \(n\)--th bounce, the vertical component of velocity will
be \({\mv{v}_y}_n = e^{n-1}\mv{u}_y\).
