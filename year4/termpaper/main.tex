\documentclass[11pt, a4paper]{pancake-article}

\usepackage{graphicx}
\usepackage{booktabs}

\usepackage[style=nature, backend=biber]{biblatex}
\addbibresource{Bibliography.bib}

\usepackage{siunitx}

\usepackage{fontspec}
\defaultfontfeatures{Scale=MatchLowercase, Ligatures=TeX}
\setmainfont{libertinus serif}
\setsansfont{tex gyre heros}
\setmonofont{monaspace neon var}
% \setmonofont{source code pro}

\usepackage{setspace}
\onehalfspacing

\usepackage[dvipsnames]{xcolor}
\newcommand{\tint}[1]{\textcolor{accent}{#1}}

\usepackage{hyperref}
\hypersetup{
  hidelinks,
  colorlinks,
  breaklinks=true,
  urlcolor=accent,
  citecolor=accent,
  linkcolor=accent,
  bookmarksopen=false,
  pdftitle={SMTP4 Geography Term Paper},
  pdfauthor={Darren Yap},
}

\usepackage[abbreviations,automake,nomain]{glossaries-extra}
\makeglossaries
\newabbreviation{we}{WE}{wind energy}
\newabbreviation{wt}{WT}{wind turbines}
\newabbreviation{wf}{WF}{wind farms}
\newabbreviation{ff}{FF}{fossil fuels}
\newabbreviation{re}{RE}{renewable energy}

\title{Up, up and away: \tint{Wind energy} and China's adoption}
\author{Darren Yap}
\date{\scshape Year Four Term Paper}

% start the document!
\begin{document}

\thispagestyle{empty}
\maketitle
\tableofcontents

\section{Introduction}
\Glsxtrfull{we} is often considered one of the cleanest ways to produce
electricity, being cheap and eco-friendly \cite{premalatha_2014_wind}.
This leads many countries to increase WE investments to counter the
rapid depletion of \glsxtrfull{ff}, an energy source widely known
to be non-renewable.
By erecting turbines in favourable areas with strong winds, \glsxtrfull{wf}
are established: the mechanical energy
winds carry turn the blades of the \glsxtrfull{wt}, after which
it is converted to electrical energy sent to a power plant for
consumption \cite{nejad_2022_wind}. \glsxtrshort{wt} are distributed over a large
land area interconnected with said power collection system. Because
\glsxtrshort{we} generated correlates with wind speed, onshore
\glsxtrshort{wf} are most
prevalent near coasts \cite{davis_2023_the}, where wind speeds are the highest.

\section{On wind energy}
China is becoming increasingly aware of the environmental detriments
\glsxtrlong{ff} bring.
International carbon dioxide emissions already reside at 35 billion
megatonnes today \cite{trndle_2020_tradeoffs},
with \glsxtrshort{ff} exploitation comprising 48 per cent of China's
emissions \cite{zhang_2012_large}---a
large margin that requires reduction.
No doubt continued \glsxtrshort{ff} exploitation hastens global warming
and endangers
natural and human life. Because \glsxtrshort{we} produces near zero
carbon emissions,
environmentally conscious governments and companies have turned to it
to reduce their
emissions, slowing global warming down if done on a large scale.
In the US, 96 million megatonnes of emissions were avoided by
introducing \glsxtrshort{we} \cite{chen_2016_growing}, establishing it
as a compelling \glsxtrfull{re} source for worldwide carbon
emission reductions.

\glsxtrshort{we} has also become increasingly affordable. Modern technological
improvements to the turbine have allowed it to generate a record power
output, while its cost has dropped dramatically.

\printglossaries
\printbibliography[heading=bibintoc]

\end{document}
