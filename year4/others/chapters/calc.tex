\chapter{Calculus questions}
\section{Tricky logarithm}
Differentiate \(\ln \ab(\frac{\sqrt[3]{8x}}{\tan x})\) with respect to \(x\),
leaving your answer as a single simplified fraction.

This expression is \it{very} difficult to differentiate on its own.
We can apply a trick! We use one of the properties of logarithms,
\begin{equation*}
	\log_x \ab(\cf{a}{b}) = \log_x a - \log_x b
\end{equation*}
to reduce our fraction down to a sum of two terms. Therefore,
\begin{align*}
	\odv{}{x} \ln \ab(\frac{\sqrt[3]{8x}}{\tan x}) & = \odv{}{x} \ab[\ln \ab(\sqrt[3]{8x}) - \ln \ab(\tan x)]                                            \\
	                                               & = \odv{}{x} \ln \ab(2x^{\cf{1}{3}}) - \odv{}{x} \ln \ab(\tan x)                                     \\
	                                               & = \frac{1}{2x^{\cf{1}{3}}} \cdot \odv{}{x} 2x^{\cf{1}{3}} - \frac{1}{\tan x} \cdot \odv{}{x} \tan x \\
	                                               & = \frac{\frac{2}{3}x^{\cf{-2}{3}}}{2x^{\cf{1}{3}}} - \frac{\sec^2 x}{\tan x}                        \\
	                                               & = \frac{x^{\cf{-2}{3}}}{3x^{\cf{1}{3}}} - \frac{\sec^2 x}{\tan x}                                   \\
	                                               & = \frac{x^{\cf{-2}{3}} \tan x - 3 \sec^2 x}{3x^{\cf{1}{3}} \sec^2 x}
\end{align*}

\section{New function!}
Find the derivative of \(x\sqrt{1-x}\) with respect to \(x\).
Given that \(\ab(\arcsin x)' = \frac{1}{\sqrt{1-x^2}}\), also find \(\int \sqrt{1-x^2} \odif{x}\).

The derivative of \(x\sqrt{1-x^2}\) is relatively easy to find by using the product rule.
\begin{align*}
	\odv{}{x} x\sqrt{1-x^2} & = \odv{x}{x} \sqrt{1-x^2} + x \odv{\sqrt{1-x^2}}{x}      \\
	                        & = \sqrt{1-x^2} + x \cdot \frac{-2x}{2\sqrt{1-x^2}}       \\
	                        & = \frac{1-x^2}{\sqrt{1-x^2}} + \frac{-x^2}{\sqrt{1-x^2}} \\
	                        & = \frac{1-2x^2}{\sqrt{1-x^2}}
\end{align*}

Now to find the integral of \(\sqrt{1-x^2}\) w.~r.~t. \(x\). Yet the
integral can't be found with any of the techniques we've learnt so far.
Instead, let's try to make this look like the derivative of \(\arcsin x\).
\begin{align*}
	\int \sqrt{1-x^2} \odif{x} & = \frac{1}{2} \int 2\sqrt{1-x^2} \odif{x}                                                                  \\
	                           & = \frac{1}{2} \int \frac{2-2x^2}{1-x^2} \sqrt{1-x^2} \odif{x}                                              \\
	                           & = \frac{1}{2} \int \frac{2-2x^2}{\sqrt{1-x^2}} \odif{x}                                                    \\
	                           & = \frac{1}{2} \int \frac{1-2x^2}{\sqrt{1-x^2}} \odif{x} + \frac{1}{2} \int \frac{1}{\sqrt{1-x^2}} \odif{x} \\
	                           & = \frac{1}{2} \ab(x \sqrt{1-x^2} + \arcsin x) + c
\end{align*}

\section{Tedious root}
Find the equation of the normal to the curve \(y = \frac{x^2}{{\ab(1 - \sqrt{x})}^6}\)
at \(x = 4\).

We start by differentiating the function to find the gradient of the tangent at \(x = 4\).
\begin{align*}
	\odv{y}{x} & = \frac{2x \cdot {\ab(1-\sqrt{x})}^6 - x^2 \cdot 6\ab(1-\sqrt{x})^5 \cdot \ab(-\frac{1}{2} x^{-\cf{1}{2}})}{{\ab(1 - \sqrt{x})}^{12}} \\
	           & = \frac{\ab(1-\sqrt{x})^5}{\ab(1-\sqrt{x})^{12}} \ab[2x\ab(1-\sqrt{x}) + 3x^{\cf{3}{2}}]                                              \\
	           & = \frac{1}{\ab(1-\sqrt{x})^7} \ab[2x\ab(1-\sqrt{x}) + 3x^{\cf{3}{2}}]                                                                 \\
	           & = \frac{x^{\cf{3}{2}} + 2x}{\ab(1-\sqrt{x})^7}
\end{align*}
Then when \(x = 4\),
\begin{align*}
	\odv[delims-eval=.|]{y}{x}_{x = 4} & = \frac{4^{\cf{3}{2}} + 2 \cdot 4}{\ab(1-\sqrt{4})^7} \\
	                                   & = \frac{8 + 8}{\ab(1-2)^7}                            \\
	                                   & = \frac{16}{-1}                                       \\
	                                   & = -16
\end{align*}
Since the normal is perpendicular to the tangent, the product of their gradients is \(-1\).
Letting the normal be some \(y = mx + c\),
\begin{align*}
	-16 \cdot m & = -1           \\
	m           & = \frac{1}{16}
\end{align*}
On the curve, when \(x = 4\),
\begin{align*}
	y & = \frac{4^2}{{\ab(1 - \sqrt{4})}^6} \\
	  & = 16
\end{align*}
At the point \(\ab(4, 16)\),
\begin{align*}
	16 & = \frac{1}{16} \cdot 4 + c \\
	c  & = \cf{63}{4}
\end{align*}
Therefore, the equation of the normal is \hil{\(y = \cf{x}{16} + \cf{63}{4}\)}.

\section{Identity crisis}
Given that \(\cos 4\theta + 4 \cos 2\theta \equiv 8 \cos^4 \theta - 3\), evaluate
\(\int_{0}^{\cf{\pi}{8}} \cos^4 2x \odif{x}\). Leave your answer in terms of \(\pi\).

Let's start by manipulating the given identity to extract our integrand.
Substituting \(\theta \coloneq 2x\),
\begin{align*}
	\cos 4\ab(2x) + 4 \cos 2\ab(2x) & = 8 \cos^4 \ab(2x) - 3                    \\
	8 \cos^4 \ab(2x)                & = \cos 8x + 4\cos 4x + 3                  \\
	\cos^4 \ab(2x)                  & = \frac{1}{8} \ab(\cos 8x + 4\cos 4x + 3)
\end{align*}
We can replace our integrand now!
\begin{align*}
	\int_{0}^{\cf{\pi}{8}} \cos^4 2x \odif{x} & = \frac{1}{8} \int_0^{\cf{\pi}{8}} \ab(\cos 8x + 4\cos 4x + 3) \odif{x}       \\
	                                          & = \frac{1}{8} \ab[\frac{\sin 8x}{8} + \sin 4x + 3x]^{\cf{\pi}{8}}_0           \\
	                                          & = \frac{1}{8} \ab(\frac{\sin\pi}{8} + \sin\frac{\pi}{2} + \frac{3\pi}{8}) - 0 \\
	                                          & = \frac{1}{8} \ab(0 + 1 + \frac{3\pi}{8})                                     \\
	                                          & = \frac{8 + 3\pi}{64}
\end{align*}
