\chapter{Inorganic Chemistry}
\section{Pre-lesson Exercise: Inorganic Chemistry}
\subsection{Balanced equations}
\subsubsection{Problem}
Write balanced equations to account for the following observations.
\begin{enumerate}
	\item When excess \ch{NaOH \aq} is added to a solution of lead(II) nitrate, a
	      white precipitate is observed. This precipitate dissolves when dilute \ch{HNO3}
	      is added, but not when dilute \ch{HCl} is added.
	\item The addition of \ch{NaOH \aq} to a pale green-coloured solution gives a dirty
	      green precipitate, which turns brown after a few minutes.
	\item The addition of hot \ch{NaOH \aq} to a colourless solution resulted in the
	      evolution of a pungent alkaline gas.
	\item The addition of \ch{HCl \aq} to a zinc salt resulted in the effervescence
	      of a gas, which when bubbled into limewater formed a white precipitate. Upon further
	      bubbling, the white precipitate dissolved.
\end{enumerate}

\subsubsection{Excess sodium hydroxide and lead(II) nitrate}
\begin{align*}
	\ch{2 NaOH \aq{} + Pb(NO3)2 \aq{}       & -> \hil{Pb(OH)2} \sld{} + 2 NaNO3 \aq} \\
	\ch{\hil{Pb(OH)2} \sld{} + 2 HNO3 \aq{} & -> Pb(NO3)2 \aq{} + 2 H2O \lqd}        \\
\end{align*}
However,
\begin{equation*}
	\ch{\hil{Pb(OH)2} \sld{} + 2 \bf{HCl} \aq{} -> PbCl2 \sld{} + 2 H2O \lqd}        \\
\end{equation*}
\ch{PbCl2} is insoluble in water, whereas \ch{Pb(NO3)2} is \sidenote{All nitrates are!}.

\subsubsection{Sodium hydroxide and the green solution}
We can identify the cation in the \hil[green!50!black]{solution} as \ch{Fe^2+}.
\begin{equation*}
	\ch{Fe^2+ \aq{} + NaOH \aq{} -> \hil{Fe(OH)2} \sld{} + Na+ \aq} \\
\end{equation*}
Since the colour of \ch{Fe^3+} is well known to be \hil[orange!50!black]{red-brown},
the precipitate had been oxidised.
\begin{equation*}
	\ch{4 \hil{Fe(OH)2} \sld{} + 2 H2O \lqd{} + O2 \gas{} -> 4 \hil{Fe(OH)3} \sld}
\end{equation*}

\subsubsection{Hot sodium hydroxide and a pungent alkaline gas}
The pungent alkaline gas is \ch{NH3 \gas} (within the scope of the high school
syllabus). Since \ch{NaOH \aq} is a base, the colourless solution is an ammonium
salt.
\begin{equation*}
	\ch{NH4^+ \aq{} + OH- \aq{} -> \hil{NH3} \gas{} + H2O \lqd}
\end{equation*}

\subsubsection{Acid and lime}
It is well-known that limewater or \ch{Ca(OH)2 \aq} forms a white precipitate when
\ch{CO2 \gas} is bubbled into it. The zinc salt is therefore \ch{ZnCO3 \sld}, which
gives \ch{CO2 \gas} when an acid is added\sidenote{This is soluble in water.}:
\begin{align*}
	\ch{ZnCO3 \sld{} + 2 HCl \aq{} & -> ZnCl2 \aq{} + CO2 \gas{} + H2O \lqd} \\
	\ch{Ca(OH)2 \aq{} + CO2 \gas{} & -> \hil{CaCO3} \sld{} + H2O \lqd}
\end{align*}
When excess \ch{CO2 \gas} is bubbled, \ch{CaCO3 \sld} will react with it to form
\ch{Ca(HCO3)2 \aq}:
\begin{equation*}
	\ch{\hil{CaCO3} \sld{} + CO2 \gas{} + H2O \lqd{} -> Ca(HCO3)2 \aq}
\end{equation*}

\subsection{Qualitative analysis}
\subsubsection{Problem}
There are three unknown samples (\bf{A} to \bf{C}) containing one metal cation
each. A few drops of each sample were added into a test tube containing \ch{NaOH \aq}
and \ch{NH3 \aq} separately, and the results were recorded as shown (Table~\ref{tab:inorg}).
\begin{table}[htpb]
	\centering
	\begin{tabular}{c l l c}
		\toprule
		\bf{Sample} & \bf{+ \ch{NaOH \aq}} & \bf{+ \ch{NH3 \aq}} & \bf{Possible cations} \\
		\midrule
		\bf{A}      & No ppt.              & No ppt.             & ?                     \\
		\bf{B}      & No ppt.              & White ppt.          & ?                     \\
		\bf{C}      & White ppt.           & White ppt.          & ?                     \\
		\bottomrule
	\end{tabular}
	\caption{Results for samples \bf{A} to \bf{C}}
	\label{tab:inorg}
\end{table}

\subsubsection{Solution}
The completed table is Table~\ref{tab:inorg-complete}.
\begin{table}[htpb]
	\centering
	\begin{tabular}{c l l c}
		\toprule
		\bf{Sample} & \bf{+ \ch{NaOH \aq}} & \bf{+ \ch{NH3 \aq}} & \bf{Possible cations}                                 \\
		\midrule
		\bf{A}      & No ppt.              & No ppt.             & {\color{accent} \ch{Na+}}                             \\
		\bf{B}      & No ppt.              & White ppt.          & {\color{accent} idk}                                  \\
		\bf{C}      & White ppt.           & White ppt.          & {\color{accent} \ch{Al^3+}, \ch{Pb^2+} or \ch{Zn^2+}} \\
		\bottomrule
	\end{tabular}
	\caption{Completed results for samples \bf{A} to \bf{C}}
	\label{tab:inorg-complete}
\end{table}

\begin{enumerate}
	\item No distinguishing tests need to be conducted.
	\item ??
	\item Add excess \ch{NH3 \aq} to the sample of \bf{C} containing \ch{NH3}. If the white precipitate dissolves, the metal cation is \ch{Zn^2+} \sidenote{Only \ch{Zn^2+} precipitates dissolve in excess \ch{NH3}!}.
	      Otherwise, add \ch{H2SO4 \aq} to a fresh sample of \bf{C} until in excess---if a white precipitate still forms, the
	      metal cation is \ch{Pb^2+} \sidenote{\ch{PbSO4} is not soluble in water, whereas \ch{Al2(SO4)3} and \ch{ZnSO4} are.}. If both these tests produce negative results, the cation is \ch{Al^3+}.
\end{enumerate}
