\chapter{Gravitation}
\label{ch:gravitation}

\section{Jupiter's moons}
\begin{problem}
The planet Jupiter has a number of moons. Data of some of these moons are given.
Plot a graph of \(\log_{10} T\) against \(\log_{10} r\).
\end{problem}

\begin{table}[htpb]
	\footnotesize
	\sisetup{
		table-number-alignment = center,
		round-mode=figures,
		round-precision=3,
		table-auto-round=true,
	}
	\begin{tabular}{l c c c c}
		\toprule
		moon     & period \(T\) / days & mean distance from centre of Jupiter \(r\) / \unit{\giga\metre} & \(\log_{10} T\)  & \(\log_{10} r\)   \\
		\midrule
		Sinope   & \num{758}           & \num{23.7}                                                      & \num{2.88}       & \num{10.37}       \\
		Leda     & \num{239}           & \num{11.1}                                                      & \hil{\num{2.38}} & \hil{\num{10.05}} \\
		Callisto & \num{16.7}          & \num{1.88}                                                      & \hil{\num{1.22}} & \hil{\num{9.27}}  \\
		Io       & \num{1.77}          & \num{.422}                                                      & \hil{\num{.248}} & \hil{\num{8.63}}  \\
		Metis    & \num{.295}          & \num{.128}                                                      & \num{-.53}       & \num{8.11}        \\
		\bottomrule
	\end{tabular}
	\label{tab:jupitermoon}
	\caption{Data on Jupiter's moons (selection)}
\end{table}

The graph is shown in \cref{fig:moons}.

\begin{figure}[htpb]
	\centering
	\begin{tikzpicture}
		\begin{axis}[
				tuftelike,
				title={Graph of \(\log_{10} T\) against \(\log_{10} r\)},
				xlabel={\(\log_{10} r\)},
				ylabel={\(\log_{10} T\)},
				% grid style=dashed,
			]
			\addplot coordinates {(8.11,-.53)(8.63,.248)(9.27,1.22)(10.05,2.38)(10.37,2.88)};
		\end{axis}
	\end{tikzpicture}
	\label{fig:moons}
	\caption{Graph of \(\log_{10} T\) against \(\log_{10} r\)}
\end{figure}

\begin{problem}
Use the data in \cref{tab:jupitermoon} to determine Jupiter's mass.
\end{problem}

After converting the raw \(T\) and \(r\) data to SI units (in \unit{\second} and \unit{\metre} respectively)
and then filling in the columns \(\log_{10} \ab(\lf{T}{\unit{\second}})\) and \(\log_{10} \ab(\lf{r}{\unit{\metre}})\),
we get the following data:
\begin{table}[htpb]
	\footnotesize
	\sisetup{
		table-number-alignment = center,
		round-mode=figures,
		round-precision=4,
		table-auto-round=true,
	}
	\begin{tabular}{l c c}
		\toprule
		moon     & \(\log_{10} \ab(\lf{T}{\unit{\second}})\) & \(\log_{10} \ab(\lf{r}{\unit{\metre}})\) \\
		\midrule
		Sinope   & \num{7.8162}                              & \num{-7.62525}                           \\
		Leda     & \num{7.3149}                              & \num{-7.9546770}                         \\
		Callisto & \num{6.1592}                              & \num{-8.725842}                          \\
		Io       & \num{5.184487}                            & \num{-9.374687549}                       \\
		Metis    & \num{4.40633575846}                       & \num{-9.892790}                          \\
		\bottomrule
	\end{tabular}
	\label{tab:jupitermoon_si}
	\caption{Data on Jupiter's moons (selection) in SI units}
\end{table}

Let's plot a new graph with the data in \cref{tab:jupitermoon_si}.
\begin{figure}[htpb]
	\centering
	\begin{tikzpicture}
		\begin{axis}[
				tuftelike,
				title={Graph of \(\log_{10} \ab(\lf{T}{\unit{\second}})\) against \(\log_{10} \ab(\lf{r}{\unit{\metre}})\)},
				xlabel={\(\log_{10} \ab(\lf{r}{\unit{\metre}})\)},
				ylabel={\(\log_{10} \ab(\lf{T}{\unit{\second}})\)},
				% grid style=dashed,
			]
			\addplot coordinates {(-9.893,4.406)(-9.375,5.184)(-8.726,6.159)(-7.955,7.315)(-7.625,7.816)};
		\end{axis}
	\end{tikzpicture}
	\label{fig:moons_si}
	\caption{Graph of \(\log_{10} \ab(\lf{T}{\unit{\second}})\) against \(\log_{10} \ab(\lf{r}{\unit{\metre}})\)}
\end{figure}

Let the mass of Jupiter be \(M\).

Using a graphing calculator, the equation of the line in \cref{fig:moons} was found to be
\begin{equation}
	\label{eq:linreg}
	\lg T = \num{1.5027}\cdot\lg r + \num{19.27173}
\end{equation}
The period of each of Jupiter's moons is controlled by the centripetal
force they experience keeping them in orbit. This centripetal force is provided
by the gravitational force of attraction between the moon and Jupiter. We can write
\begin{align*}
	\f{GM\cancel{m}}{r^2} & = \cancel{m}r\omega^2 = \f{4\pi^2\cancel{m}r}{T^2} \\
	T^2                   & = \f{4\pi^2}{GM} r^3
\end{align*}
We can linearise this:
\begin{equation}
	\label{eq:linearisation}
	\lg T = \f{3}{2} \lg r - \f{1}{2}\lg{M} + \f{1}{2}\lg{\f{4\pi^2}{G}}
\end{equation}
So we obtain
\begin{align*}
	\f{1}{2}\ab(\lg{\f{4\pi^2}{G}} - \lg{M}) & = \num{19.27173} \\
\end{align*}

\section{Binary stars}
\begin{problem}
In a binary star system, two stars \(A\) and \(B\) follow
circular orbits of radii \(R\) and \(r\) respectively about their
common centre of mass \(O\). The masses of stars \(A\) and \(B\) are \(M\) and \(m\) respectively.
Show that the period of rotation of star \(A\) is equal to that of star \(B\).
\end{problem}
The \hil{gravitational force} acting on \(B\) due to \(A\) provides
the centripetal acceleration required for \(B\) to follow its circular orbit; this holds with
respect to star \(A\) also.
Since the magnitude of this gravitational force \(F_g = \f{GMm}{\ab(R+r)^2}\) is equal
for both stars, so too is the magnitude of the centripetal force (\(\lf{4\pi^2 MR}{{T_A}^2}\) for \(A\), \(\lf{4\pi^2 mr}{T_B^2}\) for \(B\))
acting on either star.

Therefore
\begin{equation}
	\label{eq:equalperiod}
	\f{\cancel{4\pi^2}MR}{T_A^2} = \f{\cancel{4\pi^2}mr}{T_B^2}
\end{equation}

We knew from before that the centre of mass \(O\) is situated between \(A\) and \(B\).
By treating \(O\) as a pivot to a massless rod on which \(A\) and \(B\) balance, we can use the \term{principle of moments}
and say that \(mr = MR\): the anticlockwise moment contributed by \(M\) is equal to the clockwise moment contributed by \(m\).
\begin{equation}
	\label{eq:com}
	\f{R}{r} = \f{m}{M}
\end{equation}
Substituting \cref{eq:com} back into \cref{eq:equalperiod}, we get that \(T_A^2 = T_B^2\quad\Rightarrow T_A = T_B\quad\blacksquare\).

\begin{problem}
Show that the period \(T\) of rotation of the stars is given by
\begin{equation*}
	T^2 = \f{4\pi^2 \ab(R+r)^3}{G\ab(M+m)}
\end{equation*}
\end{problem}
With respect to \(A\), the gravitational force exerted on it by \(B\) provides its centripetal acceleration.
\begin{equation}
	\label{eq:fgfca}
	\f{G\cancel{M}m}{\ab(R+r)^2}=\f{4\pi^2 R\cancel{M}}{T^2}
\end{equation}
This also happens for \(B\).
\begin{equation}
	\label{eq:fgfcb}
	\f{GM\cancel{m}}{\ab(R+r)^2}=\f{4\pi^2 r\cancel{m}}{T^2}
\end{equation}
Adding \cref{eq:fgfca,eq:fgfcb}, we obtain
\begin{equation*}
	\f{G}{\ab(R+r)^2}\cdot\ab(M+m) = \f{4\pi^2}{T^2} \cdot \ab(R+r)
\end{equation*}
This leads to our \textsc{glorious} conclusion:
\begin{equation}
	\label{eq:proven}
	T^2 = \f{4\pi^2 \ab(R+r)^3}{G\ab(M+m)}\quad\blacksquare
\end{equation}
\begin{problem}
For a given binary star system, observations give the period as \(T = \qty{3.42e5}{\second}\)
and the magnitude of the velocity of one star relative to the other as \(\Delta v = \qty{2.26e5}{\metre\per\second}\).
Calculate the total mass of the binary star system.
\end{problem}
We first dissect the meaning of \hil{relative velocity}. With stars \(A\) and \(B\) to illustrate,
\begin{equation*}
	\Delta\mv{v} = \mv{v}_A - \mv{v}_B
\end{equation*}
But since \(A\) and \(B\) have tangential velocities in opposite directions,
\begin{equation*}
	\Delta {v} = v_A - \ab(-v_B) = v_A + v_B
\end{equation*}
For circular motion, \(v = r\omega = \lf{2\pi r}{T}\).
Since \(T\) is identical for both stars,
\begin{equation}
	\label{eq:relv}
	\Delta {v} = v_A + v_B = \ab(R+r)\cdot\f{2\pi}{T}
\end{equation}
We can substitute \cref{eq:relv} into \cref{eq:proven} from previously:
\begin{align*}
	m+M & = \f{\ab(R+r)^3\omega^2}{G}                   \\
	    & = \f{\omega^2}{G}\ab(\f{\Delta{v}}{\omega})^3 \\
	    & = \f{\ab(\Delta{v})^3}{G\omega}               \\
	    & = \f{T\ab(\Delta{v})^3}{2\pi G}               \\
	    & = \hil{\qty{9.42e30}{\kg}}
\end{align*}

\section{Oily planet}
\begin{problem}
A planet consists of a solid core of radius \(R\) covered uniformly with a thick
layer of fluid of thickness \(\lf{R}{2}\). The density of the fluid is \(\rho\) and
that of the solid core is \(2\rho\). Write down the gravitational field strength at \(P\)
and \(Q\) respectively in terms of \(G\), \(R\) and \(\rho\). Sketch the variation of \(g\) with the distance \(r\) from the planet's centre.
\end{problem}
Recall that \(m = \rho V\) for any object of mass \(m\), volume \(V\) and density \(\rho\).

We know that the mass of the spherical \it{solid} region is
\begin{equation*}
	m_s = \f{4}{3}\pi R^3 \cdot 2\rho = \f{8}{3}\pi\rho R^3
\end{equation*}
and the mass of the \it{liquid} region is
\begin{equation*}
	m_l = \f{4}{3} \pi\rho\ab[\ab(\lf{3R}{2})^3 - R^3] = \f{19}{6}\pi\rho R^3
\end{equation*}
Knowing that the gravitational field strength \(g\) a distance of \(r\) away from a point mass \(m\) is given by \(g = \lf{Gm}{r^2}\),
\begin{align}
	\label{eq:gfp}
	g_P & = \f{\lf{8\pi\rho R^3}{3}}{R^2} = \lf{8G\pi\rho R}{3}                                                       \\
	\label{eq:gfq}
	g_Q & = \f{m_s + m_l}{\ab(\lf{3R}{2})^2} = \f{4G}{9R^2}\ab(\f{8}{3}+\f{19}{6})\pi\rho R^3 = \lf{70G\pi\rho R}{27}
\end{align}
Outside of the planet, we know that \(g \propto \lf{1}{r^2}\). The graph is as follows:
\begin{figure}[htpb]
	\begin{tikzpicture}[
			declare function={
					gwrtr(\x) = (\x<=10) * ((1.161)*\x) + and(\x>10, \x<=15) * (\x) + (\x>15) * ((0.2)/(\x*\x));
				}
		]
		\begin{axis}[
				plainplot,
				title={Graph of \(g\) against \(r\)},
				xlabel={\(r\)},ylabel={\(g\)},
				xmin=0,xmax=25,
				%ymin=0,%ymax=100,
			]
			\pgfplotsinvokeforeach{10, 15}{
				\draw[dashed] ({rel axis cs: 0,0} -| {axis cs: #1, 0}) -- ({rel axis cs: 0,1} -| {axis cs: #1, 0});
			}
			\addplot[smooth] {gwrtr(x)};
		\end{axis}
	\end{tikzpicture}
\end{figure}


\section{Spring balances and the Earth}
\begin{problem}
The Earth may be considered to be a uniform sphere of radius \(R = \qty{6370}{\km}\),
spinning on its axis with a period of \(T = \qty{24.0}{\hour}\). The gravitational field
at the Earth's surface is identical with that of a point mass of \(M = \qty{5.98e24}{\kilogram}\)
at the centre of the Earth. For a \(m = \qty{1.00}{\kilogram}\) mass at the Equator,
\begin{itemize}
	\item calculate the gravitational force acting on the mass,
	\item determine the force required to maintainthe circular path of the mass,
	\item and deduce the reading on an accurate spring balance supporting the mass.
\end{itemize}
\end{problem}
Using \term{Newton's law of gravitation}, we find that
\begin{equation}
	\label{eq:massspringbalance}
	F_g = \f{GMm}{R^2} = \hil{\qty{9.83}{\newton}}
\end{equation}
The force required to maintain the circular path of the mass is the \hil{centripetal force} acting on it.
\begin{equation}
	\label{eq:masscentripetal}
	F_c = mR\omega^2 = \f{4\pi^2mR}{T^2} = \hil{\qty{3.37e-2}{\newton}}
\end{equation}
The gravitational force acting on the mass and the force exerted by the spring balance on it
provide for the centripetal acceleration. Therefore \(F_c = F_g - N\), and \(N = \hil{\qty{9.80}{\newton}}\) from \cref{eq:massspringbalance,eq:masscentripetal}.

\begin{problem}
What would the acceleration of the mass on the Earth's surface due to
\begin{itemize}
	\item the gravitational force alone?
	\item the force measured on the spring balance?
\end{itemize}
\end{problem}
\hil{\qty{9.83}{\meter\per\second\squared}} and \hil{\qty{9.80}{\meter\per\second\squared}} respectively.

\begin{problem}
Suppose a mass \(m\) is hung on the spring balance, and the system is now
brought to a latitude of angle \(\theta\). Deduce a general expression for the force measured
by the spring balance in terms of \(\theta\), and define all other symbols used.
\end{problem}
Let \(G\) be the gravitational constant, \(M\) be the mass of the Earth,
\(R\) be the Earth's radius, and \(T\) be its period of rotation (a day).

At the latitude \(\theta\), the gravitational force acting on the mass still points towards
the centre of the Earth. But the Earth still rotates on its axis! Therefore, the centripetal force
still points horizontally, towards the Earth's centre of rotation.

Therefore, the \term{tension} that pulls the mass attached to the spring balance is
\begin{equation*}
	\ab|T| = \sqrt{\ab|F_c|^2 + \ab|F_g|^2 - 2 \ab|F_g|\ab|F_c|\cos\theta}
\end{equation*}
Using the expressions we got in previous parts,
\begin{align*}
	\ab|T| & = \sqrt{\ab(\f{4R\pi^2}{T^2})^2 + \ab(\f{GM}{R^2})^2 - \f{8\pi^2GM}{RT^2}\cos\theta} \\
	       & = \f{\sqrt{16\pi^2 R^6 + G^2M^2T^2 - 8\pi^2GMT^2R^3\cos\theta}}{R^2T^2}
\end{align*}


\section{Three-body problem?}
\begin{problem}
For a uniform sphere of mass \(m\), the gravitational potential at points on the surface of, or outside, the sphere is exactly the same as if the sphere were a point mass \(m\) situated at the centre of the sphere. The Sun, Earth and Moon may be assumed to be uniform spheres.
\end{problem}

\section{Rocket ship}
bi B bii A biii DE

% \setcounter{section}{7}
\section{Fine line}
\begin{problem}
Points \(E\) and \(M\) are the centres of the Earth and the Moon respectively.
On the line \(EM\) lies a point \(P\), a distance from the Earth equal to \num{0.90} of the
distance from the Earth to the Moon.
Show that the magnitude of the gravitational field strength due to the Earth is equal to
that at the moon. \(m_E = \qty{5.98e24}{\kilogram}\), \(m_M = \qty{7.33e22}{\kilogram}\) and \(\ab|EP| = \qty{3.84e10}{\metre}\).
\end{problem}

\begin{align*}
	\ab|g_P| & = \f{Gm_E}{\ab|EP|^2} - \f{Gm_M}{\ab|MP|^2}                       \\
	         & = \f{G}{\ab|EM|^2} \ab[\f{m_E}{\ab(0.9)^2} - \f{m_M}{\ab(0.1)^2}] \\
	         & = 0
\end{align*}
Since the resultant gravitational field strength is zero, and the
gravitational field strengths due the Moon and the Earth are directly opposite,
both gravitational field strengths are equal in magnitude.

\begin{problem}
Calculate the gravitational potential at \(P\).
\end{problem}
Since gravitational potential is a scalar, we just need to sum up the gravitational potentials
relative to the Earth and the Moon.
\begin{align*}
	\phi_P & = \phi_E + \phi_M                                 \\
	       & = -\f{Gm_E}{\ab|EP|} + \ab(-\f{Gm_M}{\ab|MP|})    \\
	       & = -\f{G}{\ab|EP|}\ab(\f{m_E}{0.9} + \f{m_M}{0.1}) \\
	       & = \hil{\qty{-1.28e4}{\joule\per\kilogram}}
\end{align*}

\begin{problem}
The gravitational potential at the surface of the Earth is \qty{-62.5e6}{\joule\per\kilogram} and
that at the surface of the moon is \qty{-2.81e6}{\joule\per\kilogram}. Using these facts and the value
of the gravitational potential at \(P\), sketch a graph illustrating the variation of gravitational potential along a line from the surface of the Earth to the surface of the Moon.
Label your graph with appropriate numerical values.
\end{problem}

\begin{problem}
Find the least possible kinetic energy of a space vehicle of mass \(m_S = \qty{2000}{\kilogram}\)
as it leaves the Earth's surface if it is to reach the Moon's surface. Give one reason why, in practice,
the value of the minimum kinetic energy might be different from the value you calculated.
\end{problem}
We can use the \term{principle of conservation of energy} to tackle this.
Since \(\phi_P\) is the point of maximum \(\phi\), we need to make sure the
change in gravitational energy from the surface of the Earth to \(P\) is conserved
by the change in kinetic energy we calculate.
\begin{align*}
	{U_k}_i + m_S\phi_E & = \cancelto{0}{{U_k}_f} + m_S\phi_P \\
	{U_k}_i             & = m_S\ab(\phi_P - \phi_E)           \\
	                    & = \hil{\qty{1.250e11}{\joule}}
\end{align*}

\begin{problem}
A student asks if the value of the speed required above can be calculated with \(v^2 - u^2 = 2as\), with \(v = 0\) and \(a = g_E\).
Is this right?
\end{problem}

\hil{No.} Kinematic equations (such as \(v^2 - u^2 = 2as\)) mandate that the acceleration
of the moving body is constant. Since \(\phi\) can vary with the object's distance from the Earth,
so can \(g = -\odv{\phi}/{r}\). Since \(g\) is no longer constant, kinematic equations cannot apply.


\section{Extra problem}
This is \bf{CQ-B6} from the Lecture Notes.

\begin{problem}
A space station of mass \qty{4.50e5}{\kilogram} is in a low Earth orbit at an altitude of \qty{4.15e5}{\metre}. Calculate the
energy required to bring it to a medium Earth orbit of altitude \qty{2.02e7}{\metre}.
\end{problem}

Since the space station is \it{already} in orbit, gravitational force provides the centripetal force. For any radius \(r\) of the station's orbit,
\begin{align*}
	F_g               & = F_c         \\
	\f{GMm}{r^2}      & = \f{mv^2}{r} \\
	U_k = \f{mv^2}{2} & = \f{GMm}{2r}
\end{align*}
The total energy at any orbit radius \(r\) is then the sum of the kinetic and gravitational potential energies.
\begin{equation}
	\label{eq:te_station}
	U = U_k + U_p = \f{GMm}{2r} - \f{GMm}{r} = -\f{GMm}{2r}
\end{equation}
The extra work that was done is then the change in energy from the low Earth orbit radius to the medium Earth orbit radius.
\begin{align*}
	\Delta{U} & = \f{GMm}{2}\ab(\f{1}{r_i} - \f{1}{r_f})                           \\
	          & = \f{GMm}{2}\ab(\f{1}{R + r_\text{LEO}} - \f{1}{R + r_\text{MEO}}) \\
	          & =\hil{\qty{9.868e12}{\joule}}
\end{align*}