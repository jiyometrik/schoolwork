\chapter{Electric field}

\section{Electric dipole}
\begin{problem}
In free space, two point charges \(q_1 = \qty{+12}{\nano\coulomb}\) and \(q_2 = \qty{-12}{\nano\coulomb}\) are located \(d = \qty{10}{\cm}\) apart on the \(x\)-axis
and arranged symmetrically about the \(y\)-axis. A point \(M\) is \(L = \qty{12}{\cm}\) above the origin on the \(y\)-axis.
What is the magnitude and direction of the electric field at point \(M\)?
\end{problem}
The distance from \(q_1\) and \(q_2\) to point \(M\) is \(r = \sqrt{\ab(\lf{d}{2})^2 + L^2} = \qty{13}{\cm}\).
The angle of elevation of \(q_1\) to \(M\) (or \(q_2\) to \(M\)) is \(\theta = \arctan\f{L}{\lf{d}{2}} = \ang{67.38}\).

Since \(q_1\) and \(q_2\) are of the same magnitude, the magnitude of the electric field at point \(M\) due to either point charge is the same:
\begin{equation*}
	\ab|E_1| = \ab|E_2| = \f{q_1}{4\pi\epsilon_0 r^2} = \qty{6.382e3}{\newton\per\coulomb}
\end{equation*}
Since the electric field due to \(q_1\) points away from \(q_1\) and the electric field due to \(q_2\) points towards \(q_2\), the vertical components of the electric fields cancel out, while the horizontal components add up.
The net electric field at \(M\) is
\begin{equation*}
	\ab|E_\text{net}| = 2\ab|E_1|\cos\theta = \hil{\qty{4.91e3}{\newton\per\coulomb}}
\end{equation*}
The resultant electric field at point \(M\) points parallel to the \hil{positive \(x\)-axis}.

\begin{problem}
If a third charge \(q_3 = \qty{-2}{\nano\coulomb}\) is placed at point \(M\), what is the magnitude and direction of the force on \(q_3\) due to \(q_1\) and \(q_2\)?
\end{problem}
Again, we can find the vector sum of the forces due to \(q_1\) and \(q_2\) on \(q_3\).
The magnitude of the force on \(q_3\) due to either point charge is
\begin{equation*}
	\ab|F_1| = \ab|F_2| = \f{q_1 q_3}{4\pi\epsilon_0 r^2} = \qty{1.277e-5}{\newton}
\end{equation*}
Using the same reasoning as before, the net force on \(q_3\) is
\begin{equation*}
	\ab|F_\text{net}| = 2\ab|F_1|\cos\theta = \hil{\qty{9.82e-6}{\newton}}
\end{equation*}

\begin{problem}
The charges \(q_1\) and \(q_2\) form an electric dipole, since they have equal magnitude but opposite signs,
and are separated by a constant distance \(d\). Point \(M\) is located a distance \(L\) above the origin.
How does the electric field strength at point \(M\) vary with distance \(L\) when \(L\) is large?
\end{problem}
From before, the net electric field at point \(M\) is
\begin{equation}
	\label{eq:efield-dipole}
	\ab|E_\text{net}| = 2\ab|E_1|\cdot\cos\theta = \f{2q}{4\pi\epsilon_0 r^2}\cdot\f{\lf{d}{2}}{r} = \f{q}{4\pi\epsilon_0}\f{d}{r^3}
\end{equation}

When \(L \gg d\), \(\lf{d}{L} \ll 1\).
Our aim now is to determine the behaviour of \(\lf{d}{r^3}\) under this circumstance.
We can use Taylor series to help us.
\begin{align*}
	\f{d}{r^3} & = d\ab(L^2 + \f{d^2}{4})^{-\lf{3}{2}}         \\
	           & = dL^{-3}\ab(1 + \f{d^2}{4L^2})^{-\lf{3}{2}}  \\
	           & \approx dL^{-3}\ab(1 - \f{3}{2}\f{d^2}{4L^2}) \\
	           & \approx dL^{-3}
\end{align*}

Substituting this result back into \cref{eq:efield-dipole}, we have
\begin{equation*}
	\ab|E_\text{net}| \approx \f{q}{4\pi\epsilon_0}\f{d}{L^3}
\end{equation*}

\begin{problem}
Now, \(M\) lies along the positive \(x\)-axis at a distance \(L\) from the origin.
How does the electric field strength at point \(M\) vary with distance \(L\) when \(L\) is large?
\end{problem}
\(M\) is a distance \(r_1 = L + \lf{d}{2}\) from \(q_1\) and a distance \(r_2 = L - \lf{d}{2}\) from \(q_2\).
The net electric field at point \(M\) is the vector sum of the electric fields due to \(q_1\) and \(q_2\):
\begin{equation}
	\label{eq:efield-dipole-horz}
	\ab|E_\text{net}| = \ab|E_1| - \ab|E_2| = \f{q}{4\pi\epsilon_0}\ab(\f{1}{r_1^2} - \f{1}{r_2^2})
\end{equation}

Now our concern is the behaviour of \(\ab(\lf{1}{r_1^2} - \lf{1}{r_2^2})\) when \(L \gg d \iff \lf{d}{L} \ll 1\).
For \(r_1\):
\begin{align*}
	\f{1}{r_1^2} & = \ab(L + \f{d}{2})^{-2}        \\
	             & = L^{-2}\ab(1 + \f{d}{2L})^{-2} \\
	             & \approx L^{-2}\ab(1 - \f{d}{L}) \\
	             & \approx L^{-2}
\end{align*}
And for \(r_2\):
\begin{align*}
	\f{1}{r_2^2} & = \ab(L - \f{d}{2})^{-2}        \\
	             & = L^{-2}\ab(1 - \f{d}{2L})^{-2} \\
	             & \approx L^{-2}\ab(1 + \f{d}{L}) \\
	             & \approx L^{-2}
\end{align*}
Substituting back into \cref{eq:efield-dipole-horz}:
\begin{align*}
	\ab|E_\text{net}| & \approx \f{q}{4\pi\epsilon_0}\ab(L^{-2} - L^{-2}) \\
	                  & = \f{q}{4\pi\epsilon_0}\ab(0)                     \\
	                  & = \hil{0}
\end{align*}

This should not be surprising! As \(M\) moves significantly further
away from the dipole along the \(x\)-axis, \(r_1 \approx r_2\).
Since \(q_1\) and \(q_2\) are of \hil{equal magnitude and opposite charge}, the electric fields due to
either charge are also of equal magnitude and opposite direction, resulting in a net electric field of \hil{zero} at point \(M\).


\section{Electric plates}
\begin{marginfigure}
	\includegraphics[scale=.4]{assets/twoplates.png}
	\caption{Projected electron}
	\label{fig:twoplates}
\end{marginfigure}
\begin{problem}
An electron of mass \(m = \qty{9.11e-31}{\kg}\) and charge \(e = \qty{-1.60e-19}{\coulomb}\) is
projected with an initial speed \(v_0 = \qty{1.6e6}{\metre\per\second}\) into a uniform electric field
region between two parallel plates. The electric field is directed vertically downward, and
the field outside the plates is zero. Which of the plates has a higher potential?
\end{problem}
Since the electric field is directed vertically downward, the \hil{upper plate} has a higher potential.

\begin{problem}
The electron enters the field midway between the plates with its velocity perpendicular to the field.
It takes a parabolic trajectory within the field. If the electron just misses the upper plate
as it exits the field, how much time does it spend between the plates?
\end{problem}
The electron travels a horizontal distance \(x = \qty{2.00e-2}{\metre}\) with its initial
horizontal velocity \(v_0\).\sidenote{Analogous to projectile motion in a uniform gravitational field, we can take the horizontal velocity to be constant.}
The time taken is therefore \(t = \lf{x}{v_0} = \hil{\qty{1.25e-8}{\second}}\).

\begin{problem}
By considering the acceleration of the electron, find the magnitude of the electric field strength
between the plates.
\end{problem}
Consider the vertical motion of the electron. It travels
a vertical distance \(y = \qty{5.00e-3}{\metre}\) with zero initial vertical velocity.
Thus, its vertical acceleration \(a = \lf{2y}{t^2}\).

Since this acceleration is caused solely by the electric force due to the electric field,
the electric force is \(F = ma = eE\). Therefore,
\begin{align*}
	E & = \f{ma}{e}                               \\
	  & = \f{2my}{et^2}                           \\
	  & = \hil{\qty{3.64e2}{\newton\per\coulomb}}
\end{align*}

\begin{problem}
What is the change in the kinetic energy and in the electrical potential energy of the electron?
\end{problem}
The change in kinetic energy depends on the initial and final speeds of the electron.
\begin{align*}
	\Delta U_k & = \f{1}{2}m\ab(v_f^2 - {v_0}^2)            \\
	           & = \f{m}{2}\ab[v_0^2 + \ab(at)^2 - {v_0}^2] \\
	           & = \f{ma^2t^2}{2}                           \\
	           & = \hil{\qty{2.92e-19}{\joule}}
\end{align*}
We can relate the change in electric potential energy to the potential difference.
\begin{align*}
	\Delta U_p & = e\Delta V                     \\
	           & = eE\Delta y                    \\
	           & = \hil{\qty{-2.92e-19}{\joule}}
\end{align*}

\section{Electric potential}
\begin{problem}
Two point charges \(q_1 = \qty{12}{\nano\coulomb}\) and \(q_2 = \qty{-12}{\nano\coulomb}\)
are placed \(d = \qty{10}{\centi\metre}\) apart on the \(x\)-axis symmetrically about the origin.
Point \(N\) lies on the \(x\)-axis a distance \(L = \qty{2}{\centi\metre}\) to the left of the origin.
What is the electric potential due to the two charges at \(N\)?
\end{problem}
Electric potential is a scalar quantity, so we can add the sum of the electric potentials due to \(q_1\) and \(q_2\)
individually.
\begin{equation*}
	V = \f{q_1}{4\pi\epsilon_0\ab(\lf{d}{2} - L)} + \f{q_2}{4\pi\epsilon_0 \ab(\lf{d}{2}+L)} = \hil{\qty{2.05e3}{\volt}}
\end{equation*}

\begin{problem}
A third charge \(q_3 = \qty{-2}{\nano\coulomb}\) is placed at \(N\).
What is the electrical potential energy of this three-charge configuration?
\end{problem}
Electric potential energy is also a scalar quantity, so we can sum up
the electrical potential energies in between each pair of charges---\(\ab(q_1, q_2)\),
\(\ab(q_1, q_3)\) and \(\ab(q_2, q_3)\).
\begin{align*}
	U & = U_{12}+U_{13}+U_{23}                                                                       \\
	  & =\f{1}{4\pi\epsilon_0}\ab(\f{q_1q_2}{d} + \f{q_1q_3}{\lf{d}{2}-L} + \f{q_2q_3}{\lf{d}{2}+L}) \\
	  & = \hil{\qty{-1.70e-5}{\joule}}
\end{align*}


\section{Isolated nucleus}
\begin{marginfigure}
	\includegraphics[width=.85\textwidth]{assets/v-vs-r.png}
	\caption{Variation of \(V\) with \(r\)}
	\label{fig:v-vs-r}
\end{marginfigure}
An isolated nucleus in a vacuum produces an electric potential \(V\)
at a distance \(r\) from its centre, as in \cref{fig:v-vs-r}.

\begin{problem}
Use data in \cref{fig:v-vs-r} to show that there are \(54\)
protons of charge \(+e = \qty{1.60e-19}{\coulomb}\) in the nucleus.
\end{problem}
At \(r = \qty{1.0e-10}{\metre}\), \(V = \qty{780}{\volt}\)
\begin{align*}
	Q & = 4\pi\epsilon_0Vr                                      \\
	n & = \lf{Q}{e} = \lf{4\pi\epsilon_0Vr}{e} = \hil{\num{54}} \\
\end{align*}

\begin{problem}
A single proton is placed a distance of \(d = \qty{2.0e-8}{\metre}\)
from the centre of the nucleus. Suggest why it may be assumed that the proton
and nucleus behave as point charges.
\end{problem}
\hil{\(d\) is much greater than the radius of a proton.}\footnote{The radius of a proton is \qty{8.4e-16}{\metre}!}

\begin{problem}
Determine the acceleration of the proton whose mass is \(m = \qty{1.67e-27}{\kg}\) at distance \(d\).
\end{problem}
\begin{align*}
	F & = \f{Qe}{4\pi\epsilon_0d^2}                     \\
	a & = \lf{F}{m}                                     \\
	  & = \f{Qe}{4\pi\epsilon_0d^2m}                    \\
	  & = \hil{\qty{1.87e16}{\metre\per\square\second}}
\end{align*}

\begin{problem}
Determine the change \(\Delta E_k\) in kinetic energy of the proton (initially at distance \(d\))
when it has reached infinity.
\end{problem}
At infinity, the electric potential energy between two point charges tends to zero,
since the distance between them becomes very large.

Using the \term{principle of conservation of energy}:
\begin{align*}
	{E_k}_0 + U_0               & = {E_k}_f + U_f                \\
	0 + \f{Qe}{4\pi\epsilon_0d} & = {E_k}_f + 0                  \\
	\Delta{E_k}                 & = {E_k}_f - {E_k}_0            \\
	                            & = \f{Qe}{4\pi\epsilon_0d}      \\
	                            & = \hil{\qty{6.24e-19}{\joule}}
\end{align*}


\section{Charged metal spheres}
Two charged solid metal spheres \(A\) and \(B\) are situated
in a vacuum. Their centres are separated by a distance \(r = \qty{30.0}{\centi\metre}\),
as in \cref{fig:two-solid-spheres}.
\begin{marginfigure}
	\includegraphics[width=\textwidth]{assets/two-solid-spheres.png}
	\caption{Two charged solid metal spheres}
	\label{fig:two-solid-spheres}
\end{marginfigure}

Point \(P\) is a point on the line joining the centres of the spheres,
a distance \(x\) away from the centre of sphere \(A\).
\begin{figure}[htpb]
	\centering
	\includegraphics[width=0.7\textwidth]{assets/e-vs-x-two-spheres.png}
	\caption{Variation of \(E\) with \(x\)}
	\label{fig:e-vs-x}
\end{figure}


\begin{problem}
Why does \(E = 0\) for two regions of \(x\), as in \cref{fig:e-vs-x}?
\end{problem}
By definition, the electric field strength within conducting objects is zero.
Since the measurement of \(x\) includes the radii of the spheres, the regions
of \(x\) equal to the radii have \(E = 0\).

\begin{problem}
From \cref{fig:e-vs-x}, what are the radii of the spheres? Do the charges
of the spheres have the same, or the opposite, signs?
\end{problem}
We now know that the regions of \(x\) where \(E = 0\) correspond to the spheres'
radii. Hence, \hil{\(r_A = \qty{0.040}{\metre}\)}, and \hil{\(r_B = \qty{0.020}{\metre}\)}.

From \cref{fig:e-vs-x}, \(E\) is high near the spheres, and dips closer to zero
near the midpoint between the spheres. Therefore, the electric field lines
must point in opposite directions between the spheres---\hil{the spheres have the same sign}.

\begin{problem}
A lithium nucleus (\(q = \qty{4.80e-19}{\coulomb}\), \(m = \qty{1.16e-26}{\kilogram}\)) moves
along the line joining the centres of the two spheres.
\begin{itemize}
	\item Estimate the energy gained by this nucleus as it moves from \(x_0 = \qty{0.160}{\metre}\)
	      to \(x_1 = \qty{0.210}{\metre}\).
	\item Calculate the acceleration of the nucleus where \(x = \qty{.250}{\metre}\).
\end{itemize}
\end{problem}

\begin{align*}
	V & = \int_{x_0}^{x_1} E\odif{x}   \\
	  & \approx E\ab(x_1 - x_0)        \\
	  & \approx \qty{6.0e3}{\volt}     \\
	U & = qV                           \\
	  & = \hil{\qty{2.88e-15}{\joule}}
\end{align*}

The nucleus gains \qty{2.88e-15}{\joule} of energy.

At \(x_2 = \qty{.250}{\metre}\), \(E = \qty{3.0e-5}{\volt\per\metre}\).
\begin{align*}
	F & = qE                                            \\
	a & = \lf{F}{m} = \lf{qE}{m}                        \\
	  & = \hil{\qty{1.24e13}{\metre\per\square\second}}
\end{align*}